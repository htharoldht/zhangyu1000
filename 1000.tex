\documentclass[openany,twocolumn,fontset=none]{ctexbook}
\usepackage{xeCJKfntef}
\usepackage{amsmath,mathtools,unicode-math}
\setmathfont[partial=upright,StylisticSet=8]{XITS Math}
\makeatletter
\def\vdots@i#1#2#3{\vbox{
  #1\baselineskip#2\p@ \lineskiplimit\z@
  \kern#3\p@\hbox{.}\hbox{.}\hbox{.}}}
\DeclareRobustCommand\vdots{
  \mathchoice
	{\vdots@i{}{4}{6}}
	{\vdots@i{}{4}{6}}
	{\vdots@i{\scriptsize}{2}{1}}
	{\vdots@i{\tiny}{2}{1}}
}
\makeatother
\usepackage{physics,siunitx,lastpage,graphicx}
\numberwithin{figure}{section}
\renewcommand\thefigure{\arabic{chapter}-\arabic{section}-\arabic{figure}}
\usepackage{floatrow,subcaption}
\renewcommand\thesubfigure{(\alph{subfigure})}
\captionsetup[sub]{labelformat=simple}
\xeCJKDeclareCharClass{FullRight}{"2236}
\newcommand\ratio[2]{#1^^^^2236#2}

\usepackage[a4paper,top=2.5cm,bottom=2.5cm,inner=1.5cm,outer=3cm]{geometry}
\usepackage[toc]{multitoc}

\usepackage{tikz}
\usetikzlibrary{shapes.geometric,calc}
\newcommand*{\circled}[1]{\lower.7ex\hbox{\tikz\draw (0pt, 0pt)%
	circle (.5em) node {\makebox[1em][c]{\small #1}};}}
\let\libcirc\circled

\makeatletter
\newcommand{\rmnum}[1]{\romannumeral #1}
\newcommand{\Rmnum}[1]{\expandafter\@slowromancap\romannumeral #1@}
\makeatother

\usepackage{wallpaper}
\renewcommand{\CenterWallPaper}[2]{%
\AddToShipoutPicture{\put(\LenToUnit{\wpXoffset},\LenToUnit{\wpYoffset}){%
	 \parbox[b][\paperheight]{\paperwidth}{%
		\vfill
		\centering
		\tikz[opacity=0.075] \node[inner sep=0pt] {\includegraphics[angle=90,width=#1\paperwidth,height=#1\paperheight,keepaspectratio]{#2}};%
		\vfill
	 }}
  }
}

\usepackage{tabularx,diagbox}

\setlength{\headheight}{13pt}
\makeatletter
\usepackage{fancyhdr}
\pagestyle{fancy}
\fancyhf{}
\fancyhead[LO]{\bfseries \rightmark}
\fancyhead[RE]{\bfseries \leftmark}
\fancyhead[C]{\bfseries 仅供学习使用,严禁商业使用}
\fancyfoot[C]{\zihao{-5} {\kaishu 不论一个人的数学水平有多高,只要对数学拥有一颗真诚的心,他就在自己的心灵上得到了升华。}——{\itshape SCIbird}}
\fancyhead[LE,RO]{\bfseries --\thepage/\pageref{LastPage}--}
\makeatother

\usepackage{caption}
\captionsetup{labelsep=space}


\usepackage{theorem}
\ctexset{
	chapter={
		name={},
		number=0\arabic{chapter},
	},
	section={
		format={\zihao{4}\bfseries\centering},
		name={第,章},
		aftername={\hspace{1em}},
		number=\chinese{section},
	},
	subsection={
		format={\zihao{-4}\bfseries\raggedright},
		name={,、},
		aftername={\hspace{0bp}},
		number=\chinese{subsection},
	},
	subsubsection={
		format={\zihao{-4}\bfseries\raggedright},
		name={},
		aftername={\hspace{5bp}},
		number={\arabic{section}.\arabic{subsection}.\arabic{subsubsection}},
	},
}

{
	\theoremstyle{change}
	\theoremheaderfont{\bfseries}
	\theorembodyfont{\normalfont}
	\newtheorem{ti}{}[section]
}
\renewcommand{\theti}{\arabic{section}.\arabic{ti}}

{
	\theoremstyle{change}
	\theoremheaderfont{\bfseries}
	\theorembodyfont{\normalfont}
	\newtheorem{titwo}{}[chapter]
}
\renewcommand{\thetitwo}{\arabic{titwo}.}

\def\htwo{\CJKunderline*[hidden = true]{瞻彼阕者虚室生白}}
\def\kuo{ \mbox{(\hspace{1pc})}}

\usepackage{calc,ifthen}

\newlength{\choicelengtha}
\newlength{\choicelengthb}
\newlength{\choicelengthc}
\newlength{\choicelengthd}
\newlength{\choicelengthe}
\newlength{\maxlength}

\makeatletter
\newcommand{\fourch}[4]{
  \par
  \settowidth{\choicelengtha}{A.#1}
  \settowidth{\choicelengthb}{B.#2}
  \settowidth{\choicelengthc}{C.#3}
  \settowidth{\choicelengthd}{D.#4}
  \ifthenelse{\lengthtest{\choicelengtha>\choicelengthb}}{\setlength{\maxlength}{\choicelengtha}}{\setlength{\maxlength}{\choicelengthb}}
  \ifthenelse{\lengthtest{\choicelengthc>\maxlength}}{\setlength{\maxlength}{\choicelengthc}}{}
  \ifthenelse{\lengthtest{\choicelengthd>\maxlength}}{\setlength{\maxlength}{\choicelengthd}}{}
  \ifthenelse{\lengthtest{\maxlength>0.48\linewidth}}
  {%
    \noindent%
    \begin{tabular}{@{}p{\linewidth}@{}}
      \setlength\tabcolsep{0pt}
      \@hangfrom{\textsf A.}#1 \\
      \@hangfrom{\textsf B.}#2 \\
      \@hangfrom{\textsf C.}#3 \\
      \@hangfrom{\textsf D.}#4 \\
    \end{tabular}
  }%
  {%
    \ifthenelse{\lengthtest{\maxlength>0.22\linewidth}}
    {%
      \noindent%
      \begin{tabular}{@{}p{0.48\linewidth}@{\hspace*{0.04\linewidth}}p{0.48\linewidth}@{}}
        \setlength\tabcolsep{0pt}
        \@hangfrom{\textsf A.}#1 & \@hangfrom{\textsf B.}#2 \\
        \@hangfrom{\textsf C.}#3 & \@hangfrom{\textsf D.}#4 \\
      \end{tabular}
    }%
    {%
      \noindent%
      \begin{tabular}{@{}*{3}{p{0.22\linewidth}@{\hspace*{0.04\linewidth}}}p{0.22\linewidth}@{}}
        \setlength\tabcolsep{0pt}
        \@hangfrom{\textsf A.}#1  & \@hangfrom{\textsf B.}#2 & \@hangfrom{\textsf C.}#3 & \@hangfrom{\textsf D.}#4 \\
      \end{tabular}
    }%
  }%
  \unskip\unskip
}
\makeatother
\let\twoch=\fourch
\let\onech=\fourch

\let\bm=\symbfit
\let\mathrm=\symup
\let\mathbb=\symbb
\def\leq{\leqslant}
\def\geq{\geqslant}
\def\ee{\mathrm{e}}
\def\CC{\mathrm{C}}
\def\TT{\mathrm{T}}
\def\AA{\mathrm{A}}
\def\astt{*}
\let\mylim=\lim
\def\lim{\mylim\limits}
\let\mysum=\sum
\def\sum{\mysum\limits}
\let\div\relax
\DeclareMathOperator{\div}{div}
\let\grad\relax
\DeclareMathOperator{\grad}{grad}
\DeclareMathOperator{\rot}{rot}
\DeclareMathOperator{\Cov}{Cov}
\long\def\guanggao{
	\vspace*{\fill}
	\begin{center}
		\bfseries 广告位招租
	\end{center}
	\vspace*{\fill}
}
\def\theenumi{\arabic{enumi}}
\def\labelenumi{(\theenumi)}

\setmainfont{XITS}
\defaultCJKfontfeatures{Mapping=fullwidth-stop}
\IfFontExistsTF{Source Han Serif SC}{
  \setCJKmainfont{Source Han Serif SC}[
    UprightFont     = *-Regular,
    BoldFont        = *-Bold,
    ItalicFont      = 方正新楷体简体,
    BoldItalicFont  = *-Bold
  ]
  \setCJKsansfont{Source Han Sans SC}[
    UprightFont     = *-Regular,
    BoldFont        = *-Bold,
    BoldItalicFont  = *-Bold
  ]
	\setCJKmonofont{仿宋}
	\newCJKfontfamily[kaishu]\kaishu{方正新楷体简体}
}{
  \ctexset{fontset=fandol}
}


\usepackage[bookmarksopen=true,bookmarksnumbered=true,hidelinks,pdftitle=2020张宇1000题(数一),pdfauthor=张宇]{hyperref}

\title{《考研数学题源探析经典 1000 题》(习题分册·数学一)\\\LaTeX{}重排版\thanks{Build time:\today,Releases:v1.1}}
\author{张宇}
\date{2019 年 3 月}


\begin{document}
	\frontmatter
	\maketitle
	\input{chapter/chap0.tex}
	\tableofcontents
	\mainmatter
	\chapter{高等数学}
	高等数学是硕士研究生招生考试考查内容之一,主要考查考生对高等数学的基本概念、基本理论、基本方法的理解和掌握以及考生的抽象思维能力、逻辑推理能力、综合运用能力和解决实际问题的能力。在考研数学一试卷中分值为82分,约占 \SI{56}{\percent}。
	\input{chapter/sec1-1.tex}
	\subsection{无穷小比阶}

	\begin{ti}
		当 $x \to 0$ ,$(1 - \cos x)\ln\left( 1 + 2x^{3} \right)$ 是比 $x \sin x^{n}$ 高阶的无穷小,而 $x \sin x^{n}$ 是比 $\ee^{x\tan^{2} x} - 1$ 高阶的无穷小,则正整数 $n = $ \htwo.
	\end{ti}

	\begin{ti}
		当 $x \to 0^{+}$ 时,$\sqrt{1 + \tan \sqrt{x}} - \sqrt{1 + \sin\sqrt{x}}$ 是 $x$ 的 $k$ 阶无穷小,则 $k =$ \htwo.
	\end{ti}

	\begin{ti}
		当 $x \to 0$ 时,$f(x) = \ln\left( 1+x^{2} \right) - 2\sqrt[3]{\left( \ee^{x} - 1 \right)^{2}}$ 是无穷小量 $x^{k}$ 的同阶无穷小,则 $k = $ \kuo.

		\fourch{$1$}{$2$}{$\frac{2}{3}$}{$\frac{3}{2}$}
	\end{ti}

	\begin{ti}
		当 $x \to 0$ 时,下列无穷小量中,最高阶的无穷小是\kuo.

		\twoch{$\ln\left( x + \sqrt{1 + x^{2}} \right)$}{$1 - \cos x$}{$\tan x - \sin x$}{$\ee^{x} + \ee^{-x} - 2$}
	\end{ti}

	\begin{ti}
		当 $x \to 0^{+}$ 时,下列无穷小量中,与 $x$ 同阶的无穷小是\kuo.

		\twoch{$\sqrt{1 + x} - 1$}{$\ln(1 + x) - x$}{$\cos(\sin x) - 1$}{$x^{x} - 1$}
	\end{ti}

	\begin{ti}
		当 $x \to 0$ 时,$f(x) = x - \sin x + \int_{0}^{x} t^{2} \ee^{t^{2}} \dd{t}$ 是 $x$ 的 $k$ 阶无穷小,则 $k=$ \kuo.

		\fourch{$3$}{$4$}{$5$}{$6$}
	\end{ti}

	\begin{ti}
		当 $x \to 0^{+}$ 时,试比较无穷小量 $\alpha$,$\beta$ 和 $\gamma$ 三者之间的阶,其中
		\[
			\alpha = \int_{0}^{x} \cos t^{2} \dd{t},\beta = \int_{0}^{x^{2}} \tan \sqrt{t} \dd{t},\gamma = \int_{0}^{\sqrt{x}} \sin t^{3} \dd{t}.
		\]
	\end{ti}

	\begin{ti}
		当 $x \to 0$ 时,$\sin x \left( \cos x - 4 \right) + 3x$ 为 $x$ 的几阶无穷小?
	\end{ti}

	\begin{ti}
		当 $x \to 0$ 时,确定下列无穷小量的阶数:
		\begin{enumerate}
			\item $\tan\left( \sqrt{x+2} - \sqrt{2} \right)$;
			\item $\sqrt[3]{1 + \sqrt[3]{x}} - 1$;
			\item $3^{\sqrt{x}} - 1$.  
		\end{enumerate}
	\end{ti}

	\begin{ti}
		当 $x \to 0$ 时,$x - \sin x \cos x \cos 2x$ 与 $cx^{k}$ 为等价无穷小,则 $c=$ \htwo,$k=$ \htwo.
	\end{ti}

	\begin{ti}
		当 $x \to 0$ 时,$1 - \cos x \cos 2x \cos 3x$ 对于无穷小 $x$ 的阶数等于 \htwo.
	\end{ti}

	\begin{ti}
		极限 $\lim_{x \to \infty} \frac{\ee^{\sin\frac{1}{x}}-1}{\left( 1 + \frac{1}{x} \right)^{\alpha} - \left( 1 + \frac{1}{x} \right)} = A \ne 0$ 的充要条件是\kuo.

		\twoch{$\alpha > 1$}{$\alpha \ne 1$}{$\alpha > 0$}{与 $\alpha$ 无关}
	\end{ti}

	\begin{ti}
		设当 $x \to 0$ 时,$\ee^{\tan x} - \ee^{x}$ 与 $x^{n}$ 是同阶无穷小,则 $n$ 为 \kuo.

		\fourch{$1$}{$2$}{$3$}{$4$}
	\end{ti}

	\begin{ti}
		设当 $x \to 0$ 时,$f(x) = ax^{3} + bx$ 与 $g(x) =$ $\int_{0}^{\sin x} \left( \ee^{t^{2}} -1 \right) \dd{t}$ 是等价无穷小,则\kuo.

		\twoch{$a = \frac{1}{3},b=1$}{$a = 3,b=0$}{$a = \frac{1}{3},b=0$}{$a = 1,b=0$}
	\end{ti}

	\begin{ti}
		设当 $x \to 0$ 时,$f(x) = \ln\left( 1+x^{2} \right) - \ln\bigl( 1 + \sin^{2}x \bigr)$ 是 $x$ 的 $n$ 阶无穷小,则正整数 $n$ 为\kuo.
		
		\fourch{$1$}{$2$}{$3$}{$4$}
	\end{ti}

	\begin{ti}
		当 $x \to \uppi$ 时,若有 $\sqrt[4]{\sin\frac{x}{2}} - 1 \sim A(x - \uppi)^{k}$,则 $A=$\htwo,$k=$\htwo.
	\end{ti}

	\begin{ti}
		半径分别为 $R,r(R>r>0)$ 的两个圆相切于坐标轴原点. 如图~\ref{fig:1.1.1} 所示.
		\begin{enumerate}
			\item 当 $x \to 0^{+}$ 时,若线段长 $MM_{1}$ 与 $x^{k}$ 同阶,求 $k$;
			\item 当 $x \to 0^{+}$ 时,若 $\angle MOM_{1}$ 与 $x^{c}$ 同阶,求 $c$.
		\end{enumerate}
		\begin{figure}[htbp]
			\centering
			\includegraphics[scale=1]{figure/fig1-1-1.pdf}
			\caption{}\label{fig:1.1.1}
		\end{figure}
	\end{ti}
	\input{chapter/sec1-3.tex}
	\subsection{连续与间断}

	\begin{ti}
		当 $x \in \left( -\frac{1}{2},1 \right]$ 时,确定函数 $f(x) = \frac{\tan \uppi x}{|x|\left( x^{2} - 1 \right)}$ 的间断点,并判定其类型.
	\end{ti}

	\begin{ti}
		确定函数 $f(x) = \frac{x(x - 1)}{|x| x^{2} - |x|}$ 的间断点,并判定其类型.
	\end{ti}

	\begin{ti}
		设 $a > 0$,$b > 0$,$c > 0$,
		\[
			A(x) = \begin{cases}
				\left( \frac{a^{x} + b^{x}}{2} \right)^{\frac{1}{x}}, & x \ne 0,\\
				c, & x = 0.
			\end{cases}
		\]
		\begin{enumerate}
			\item 讨论 $A(x)$ 在 $x = 0$ 处的连续性;
			\item 讨论 $\lim_{x \to +\infty} A(x)$,$\lim_{x \to -\infty} A(x)$,$\lim_{x \to 0} A(x)$,$A(-1)$,$A(1)$ 五者之间的大小关系.
		\end{enumerate}
	\end{ti}

	\begin{ti}
		求 $f(x) = \frac{1}{1 - \ee^{\frac{x}{1 - x}}}$ 的连续区间、间断点,并判别间断点的类型.
	\end{ti}

	\begin{ti}
		求函数 $f(x) = \lim_{n \to \infty} \frac{x^{n+2} - x^{-n}}{x^{n} + x^{-n}}$ 的间断点并指出其类型.
	\end{ti}

	\begin{ti}
		若
		\[
			f(x) = \frac{\sqrt[3]{x}}{\lambda - \ee^{-kx}}
		\]
		在 $(-\infty,+\infty)$ 内连续,且 $\lim_{x \to -\infty} f(x) = 0$,则\kuo.
		
		\twoch{$\lambda < 0, k < 0$}{$\lambda < 0, k > 0$}{$\lambda \geq 0, k < 0$}{$\lambda \leq 0, k > 0$}
	\end{ti}

	\begin{ti}
		若
		\[
			f(x) = \begin{cases}
				\ee^{x} (\sin x + \cos x), & x > 0,\\
				2x + a, & x \leq 0
			\end{cases}
		\]
		 是 $(-\infty,+\infty)$ 内的连续函数,则 $a =$\htwo.
	\end{ti}

	\begin{ti}
		试讨论函数 $g(x) = \begin{cases}
			x^{\alpha} \sin\frac{1}{x}, & x > 0,\\
			\ee^{x} + \beta, & x \leq 0
		\end{cases}$ 在点 $x = 0$ 处的连续性.
	\end{ti}

	\begin{ti}
		求函数 $F(x) = \begin{cases}
			\frac{x(\uppi + 2x)}{2 \cos x}, & x \leq 0,\\
			\sin\frac{1}{x^{2} - 1}, & x > 0
		\end{cases}$ 的间断点,并判断它们的类型.
	\end{ti}

	\begin{ti}
		设 $f(x) = \lim_{n \to \infty}\frac{\ee^{\frac{1}{x}} \arctan\frac{1}{1 + x}}{x^{2} + \ee^{nx}}$,求 $f(x)$ 的间断点并判定其类型.
	\end{ti}

	\begin{ti}
		设 $f(x) = \begin{cases}
			\ee^{\frac{1}{x - 1}}, & x > 0,\\
			\ln(1 + x), & -1 < x < 0,
		\end{cases}$ 求 $f(x)$ 的间断点,并说明间断点的类型.
	\end{ti}

	\begin{ti}
		设 $f(x;t) = \left( \frac{x - 1}{t - 1} \right)^{\frac{t}{x - t}}((x - 1)(t - 1)>0, x \ne t)$,函数 $f(x)$ 由表达式
		\[
			f(x) = \lim_{t \to x}f(x;t)
		\]
		确定,求 $f(x)$ 的连续区间和间断点,并判定间断点的类型.
	\end{ti}

	\begin{ti}
		设函数 $f(x)$ 在 $[a,b]$ 上连续,$x_{1},x_{2},\cdots,x_{n},\cdots$ 是 $[a,b]$ 上的一个点列,求 $\lim_{n \to \infty} \sqrt[n]{\frac{1}{n}\sum_{k=1}^{n}\ee^{f(x_{k})}}$.
	\end{ti}

	\begin{ti}
		\begin{enumerate}
			\item 求函数 \[f(x) = \lim_{n \to \infty} \sqrt[n]{1 + (2x)^{n} + x^{2n}}\] $(x \geq 0)$ 的表达式;
			\item 讨论函数 $f(x)$ 的连续性.
		\end{enumerate}
	\end{ti}

	\begin{ti}
		已知 $f(x) = \lim_{n \to \infty} \frac{x^{2n-1} + ax^{2} + bx}{x^{2n} + 1}$ 是连续函数,求 $a,b$ 的值.
	\end{ti}

	\begin{ti}
		求函数 $f(x) = \frac{x^{3} + 1}{|x + 1|\left( x^{2} - x \right)} \sin\left( \frac{|x - 1|}{x + 2}\uppi \right)$ 的所有间断点,并判断它们的类型.
	\end{ti}
	\section{一元函数微分学}
	\subsection{一点的导数问题}

	\begin{ti}
		设 $f(x)$ 在 $x = 1$ 处可导,$f'(1) = 1$,求 $\lim_{x \to 1} \frac{f(x) - f(1)}{x^{10} - 1}$.
	\end{ti}

	\begin{ti}
		设 $f(x)$ 在 $x = 0$ 处连续,且 \[\lim_{x \to 0} \left[ \frac{\ee^{f(x)} - \cos x + \sin x}{x} \right] = 0,\]求 $f(0)$,并讨论 $f(x)$ 在 $x = 0$ 处是否可导?若可导,请求出 $f'(0)$.
	\end{ti}

	\begin{ti}
		函数 $f(x)$ 在 $(-\infty,+\infty)$ 内有定义,在区间 $[0,2]$ 上,$f(x) = x\left( x^{2} - 4 \right)$. 假若对任意的 $x$ 都满足 $f(x) = k f(x + 2)$,其中 $k$ 为常数.
		\begin{enumerate}
			\item 写出 $f(x)$ 在 $[-2,0)$ 上的表达式;
			\item 问 $k$ 为何值时,$f(x)$ 在 $x = 0$ 处可导?
		\end{enumerate}
	\end{ti}

	\begin{ti}
		设 $f(x)$ 在 $(-\infty,+\infty)$ 内有定义,且 $f'(0) = a(a \ne 0)$,又对任意的 $x,y \in (-\infty,+\infty)$,有
		\[
			f(x + y) = \frac{f(x) + f(y)}{1 - f(x)f(y)},
		\]
		求 $f(x)$.
	\end{ti}
	
	\begin{ti}
		设 $f(x)$ 在 $(-\infty,+\infty)$ 内有定义,且对任意的 $x,x_{1},x_{2} \in (-\infty,+\infty)$,有
		\[
			f(x_{1} + x_{2}) = f(x_{1}) \cdot f(x_{2}),f(x) = 1 + xg(x),
		\]
		其中 $\lim_{x \to 0} g(x) = 1$. 证明:$f(x)$ 在 $(-\infty,+\infty)$ 内处处可导.
	\end{ti}

	\begin{ti}
		设 $f(x)$ 定义在 $\mathbb{R}$ 上,对于任意的 $x_{1},x_{2}$,有 $|f(x_{1}) - f(x_{2})| \leq (x_{1} - x_{2})^{2}$,求证:$f(x)$ 是常值函数.
	\end{ti}

	\begin{ti}
		设 $f''(1)$ 存在,且 $\lim_{x \to 1}\frac{f(x)}{x - 1} = 0$. 记
		\[
			\varphi(x) = \int_{0}^{1} f'[1 + (x - 1)t]\dd{t}.
		\]
		求 $\varphi(x)$ 在 $x = 1$ 的某个邻域内的导数,并讨论 $\varphi'(x)$ 在 $x = 1$ 处的连续性.
	\end{ti}

	\begin{ti}
		设函数
		\[
			f(x) = \begin{cases}
				x^{3} \sin\frac{1}{x}, & x \ne 0,\\
				0, & x = 0.
			\end{cases}
		\]
		讨论 $f(x)$ 在 $x = 0$ 的可导性以及 $f'(x)$ 在 $x = 0$ 的连续性.
	\end{ti}

	\begin{ti}
		已知函数 $f(x) = \begin{cases}
			\frac{\int_{x}^{2x} \ee^{t^{2}} \dd{t}}{x}, & x \ne 0,\\
			a, & x = 0
		\end{cases}$ 在 $x = 0$ 处可导. 求
		\begin{enumerate}
			\item $a$ 的值;
			\item $f'(0)$.
		\end{enumerate}
	\end{ti}

	\begin{ti}
		若 $f(x) = \begin{cases}
			\ln\left( 1 + x^{2} \right), & x \leq 0,\\
			a \sin x + 2x, & x > 0
		\end{cases}$ 是可导函数,则 $a = $\htwo.
	\end{ti}

	\begin{ti}
		设 $f(x) = \begin{cases}
			\frac{1 - \cos x}{\sqrt{x}}, & x > 0,\\
			x^{2} g(x), & x \leq 0,
		\end{cases}$ 其中 $g(x)$ 是有界函数,则 $f(x)$ 在 $x = 0$ 处\kuo.

		\twoch{极限不存在}{极限存在,但不连续}{连续,但不可导}{可导}
	\end{ti}

	\begin{ti}
		设函数 $f(x)$ 是定义在 $(-1,1)$ 内的奇函数,且 $\lim_{x \to 0^{+}} \frac{f(x)}{x} = a \ne 0$,则 $f(x)$ 在 $x = 0$ 处的导数为\kuo.

		\fourch{$a$}{$-a$}{$0$}{不存在}
	\end{ti}

	\begin{ti}
		设函数 $f(x)$ 在 $x = 0$ 处连续,且 $\lim_{x \to 0} \frac{f\left( x^{2} \right)}{x^{2}} = 1$,则\kuo.

		\onech{$f(0) = 0$ 且 $f_{-}'(0)$ 存在}{$f(0) = 1$ 且 $f_{-}'(0)$ 存在}{$f(0) = 0$ 且 $f_{+}'(0)$ 存在}{$f(0) = 1$ 且 $f_{+}'(0)$ 存在}
	\end{ti}

	\begin{ti}
		设 $g(x)$ 在 $x = 0$ 处二阶可导,且 $g(0) = g'(0) = 0$,设
		\[
			f(x) = \begin{cases}
				\frac{g(x)}{x}, & x \ne 0,\\
				0, & x = 0,
			\end{cases}
		\]
		则 $f(x)$ 在 $x = 0$ 处\kuo.

		\onech{不连续}{连续,但不可导}{可导,但导函数不连续}{可导且导函数连续}
	\end{ti}

	\begin{ti}
		若
		\[
			f(x) = \ee^{10x} x (x + 1) (x + 2) \cdots (x + 10),
		\]
		则 $f'(0) = $\htwo.
	\end{ti}

	\begin{ti}
		已知 $f(x) = \frac{(x - 1) (x - 2) (x - 3) \cdots (x - 100)}{(x + 1) (x + 2) (x + 3) \cdots (x + 100)}$,求 $f'(1)$.
	\end{ti}

	\begin{ti}
		设函数 $f(x) = \left( \ee^{x} - 1 \right) \left( \ee^{2x} - 2 \right) \cdots \left( \ee^{nx} - n \right)$,其中 $n$ 为正整数,则 $f'(0) = $\kuo.

		\twoch{$(-1)^{n-1}(n - 1)!$}{$(-1)^{n}(n - 1)!$}{$(-1)^{n-1}n!$}{$(-1)^{n}n!$}
	\end{ti}

	\begin{ti}
		已知 $f(x) = \sqrt{1 + x} + \arcsin\frac{1 - x}{1 + x^{2}}$,求 $f'(1)$.
	\end{ti}

	\begin{ti}
		设 $f(x) = \sqrt{\frac{(1 + x)\sqrt{x}}{\ee^{x - 1}}} + \arcsin\frac{1 - x}{\sqrt{1 + x^{2}}}$,求 $f'(1)$.
	\end{ti}

	\begin{ti}
		设 $f(x)$ 可导,$F(x) = f(x) (1 + |\sin x|)$,若使 $F(x)$ 在 $x = 0$ 处可导,则必有\kuo.

		\twoch{$f(0) = 0$}{$f'(0) = 0$}{$f(0) + f'(0) = 0$}{$f(0) - f'(0) = 0$}
	\end{ti}

	\begin{ti}
		设 $f(x)$ 在 $x = a$ 处连续,$F(x) = f(x) |x - a|$,则 $f(a) = 0$ 是 $F(x)$ 在 $x = a$ 处可导的\kuo.

		\onech{充要条件}{充分非必要条件}{必要非充分条件}{既非充分又非必要条件}
	\end{ti}

	\begin{ti}
		函数 $F(x) = \left( x^{2} - x - 2 \right)\left| x^{3} - x \right|$ 不可导的点的个数为\kuo.
		
		\fourch{$1$}{$2$}{$3$}{$4$}
	\end{ti}

	\begin{ti}
		设 $f(x) = \left| \begin{smallmatrix}
			1 & x - 1 & 2 x - 1\\
			1 & x - 2 & 3 x - 2\\
			1 & x - 3 & 4 x - 3
		\end{smallmatrix} \right|$,证明:存在 $\xi \in (0,1)$,使得 $f'(\xi) = 0$.
	\end{ti}
	\subsection{导数计算}
	
	\begin{ti}
		设 $y = \ee^{x^{2}}$,求 $\frac{\dd{y}}{\dd{x}}, \frac{\dd{y}}{\dd{\left( x^{2} \right)}}, \frac{\dd^{2}{y}}{\dd{x^{2}}}$
	\end{ti}

	\begin{ti}
		设 $f(x) = (\cos x - 4)\sin x + 3x$.
		\begin{enumerate}
			\item 求 $\frac{\dd{f(x)}}{\dd{\left( x^{2} \right)}}$;
			\item 当 $x \to 0$ 时,$f(x)$ 为 $x$ 的几阶无穷小?
		\end{enumerate}
	\end{ti}

	\begin{ti}
		设 $f'(0) = 1$,$f''(0) = 0$,求证:在 $x = 0$ 处,有
		\[
			\frac{\dd^{2}}{\dd{x^{2}}} f\left( x^{2} \right) = \frac{\dd^{2}}{\dd{x^{2}}} f^{2}(x).
		\]
	\end{ti}

	\begin{ti}
		设 $f(x)$ 为可微函数,证明:若 $x = 1$ 时,有 $\frac{\dd{f\left( x^{2} \right)}}{\dd{x}} = \frac{\dd{f^{2}(x)}}{\dd{x}}$,则必有 $f'(1) = 0$ 或 $f(1) = 1$.
	\end{ti}
	
	\begin{ti}
		设函数 $f(x) = x^{3} + 2x - 4$,$g(x) = f[f(x)]$,则 $g'(0) =$\htwo.
	\end{ti}

	\begin{ti}
		设 $y = f\left( \frac{3x - 2}{3x + 2} \right)$ 且 $f'(x) = \arctan x^{2}$,求 $\left. \frac{\dd{y}}{\dd{x}} \right|_{x = 0}$.
	\end{ti}

	\begin{ti}
		设 $f(x) = \begin{cases}
			x^{3x}, & x > 0,\\
			x + 1, & x \leq 0,
		\end{cases}$ 求 $f''(x)$.
	\end{ti}

	\begin{ti}
		设 $f(x)$ 在 $(-\infty,+\infty)$ 内连续且大于 $0$,
		\[
			g(x) = \begin{cases}
				\frac{\int_{0}^{x} tf(t) \dd{t}}{\int_{0}^{x} f(t) \dd{t}}, & x \ne 0,\\
				0, & x = 0.
			\end{cases}
		\]
		\begin{enumerate}
			\item 求 $g'(x)$;
			\item 证明:$g'(x)$ 在 $(-\infty,+\infty)$ 内连续.
		\end{enumerate}
	\end{ti}

	\begin{ti}
		已知可微函数 $y = y(x)$ 由方程 $y = - y\ee^{x} + 2\ee^{y} \sin x - 7x$ 所确定,求 $y''(0)$.
	\end{ti}

	\begin{ti}
		设函数 $y = y(x)$ 由参数方程 $\begin{cases}
			x = 1 + t^{2},\\
			y = \cos t
		\end{cases}$ 所确定,求:
		\begin{enumerate}
			\item $\frac{\dd{y}}{\dd{x}}$ 和 $\frac{\dd^{2}y}{\dd{x^{2}}}$;
			\item $\lim_{x \to 1^{+}} \frac{\dd{y}}{\dd{x}}$ 和 $\lim_{x \to 1^{+}} \frac{\dd^{2}y}{\dd{x^{2}}}$.
		\end{enumerate}
	\end{ti}

	\begin{ti}
		设函数 $f(x)$ 二阶可导,$f'(0) = 1$,$f''(0) = 2$,且 $\begin{cases}
			x = f(t) - \uppi,\\
			y = f\left( \ee^{3t} - 1 \right),
		\end{cases}$ 求 $\left. \frac{\dd{y}}{\dd{x}} \right|_{t = 0}$,$\left. \frac{\dd^{2}y}{\dd{x^{2}}} \right|_{t = 0}$.
	\end{ti}

	\begin{ti}
		设函数 $y = f(x)$ 是由
		\[
			\begin{cases}
				x^{x} + tx - t^{2} = 0,\\
				\arctan(ty) = \ln\left( 1 + t^{2}y^{2} \right)
			\end{cases}
		\]
		确定,求 $\frac{\dd{y}}{\dd{x}}$.
	\end{ti}

	\begin{ti}
		设 $u = f\left[ \varphi(x) + y^{2} \right]$,其中 $y = y(x)$ 由方程 $y + \ee^{y} = x$ 确定,且 $f(x), \varphi(x)$ 均有二阶导数,求 $\frac{\dd{u}}{\dd{x}}$ 和 $\frac{\dd^{2}u}{\dd{x^{2}}}$.
	\end{ti}

	\begin{ti}
		设 $y = x^{3} + 3x + 1$,则 $\left. \frac{\dd{x}}{\dd{y}} \right|_{y = 1}=$\htwo.
	\end{ti}

	\begin{ti}
		设 $x = f(y)$ 是函数 $y = x + \ln x$ 的反函数,求 $\frac{\dd^{2}f}{\dd{y^{2}}}$.
	\end{ti}

	\begin{ti}
		设 $y = f(x)$ 与 $x = g(y)$ 互为反函数,$y = f(x)$ 可导,且 $f'(x) \ne 0$,$f(3) = 5$,
		\[
			h(x) = f\left[ \frac{1}{3} g^{2}\left( x^{2} + 3x + 1 \right) \right],
		\]
		求 $h'(1)$.
	\end{ti}

	\begin{ti}
		设 $y = \left[ (1 + x)(3 + x)^{9} \right]^{\frac{1}{2}} (2 + x)^{4}$,求 $y'(0)$.
	\end{ti}

	\begin{ti}
		已知 $u = g(\sin y)$,其中 $g'(v)$ 存在,$y = f(x)$ 由参数方程
		\[
			\begin{cases}
				x = a \cos t,\\
				y = b \sin t
			\end{cases}
			\left( 0 < t < \frac{\uppi}{2}, a \ne 0 \right)
		\]
		所确定,求 $\dd{u}$.
	\end{ti}

	\begin{ti}
		设 $x = f(t) \cos t - f'(t) \sin t$,$y = f(t) \sin t + f'(t) \cos t$,$f''(t)$ 存在,试证:
		\[
			(\dd{x})^{2} + (\dd{y})^{2} = \left[ f(t) + f''(t) \right]^{2} (\dd{t})^{2}.
		\]
	\end{ti}

	\begin{ti}
		设 $f(x) = x \ee^{-x}$,则 $f^{(n)}(x) = $\kuo.
		
		\twoch{$(-1)^{n} (1 + n) x \ee^{-x}$}{$(-1)^{n} (1 - n) x \ee^{-x}$}{$(-1)^{n} (x + n) \ee^{-x}$}{$(-1)^{n} (x - n) \ee^{-x}$}
	\end{ti}

	\begin{ti}
		若 $f(x) = x^{5} \ee^{6x}$,则 $f^{(2019)}(0) = $\htwo.
	\end{ti}

	\begin{ti}
		设 $f(x) = \frac{x}{1 - 2x^{4}}$,则 $f^{(101)}(0) = $\htwo.
	\end{ti}

	\begin{ti}
		设 $f(x) = \ee^{x} \sin x$,则 $f^{(7)}(x) = $\htwo.
	\end{ti}

	\begin{ti}
		设 \[f(x) = \lim_{n \to \infty} x \cos 2x \cos \frac{x}{2} \cos \frac{x}{4} \cdots \cos \frac{x}{2^{n}}(x > 0).\]
		\begin{enumerate}
			\item 求证 $f(x) = \cos 2x \sin x$;
			\item 求 $f^{(20)}(x)$.
		\end{enumerate}
	\end{ti}

	\begin{ti}
		设 $f(x) = \left( x^{2} - 3x + 2 \right)^{n} \cos \frac{\uppi x^{2}}{16}$,求 $f^{(n)}(2)$.
	\end{ti}

	\begin{ti}
		设 $y = \arcsin x$.
		\begin{enumerate}
			\item 证明其满足方程 $\left( 1 - x^{2} \right) y^{(n+2)} - (2n + 1)\* x\times y^{(n+1)} - n^{2} y^{(n)} = 0 (n \geq 0)$;
			\item 求 $\left. y^{(n)} \right|_{x = 0}$.
		\end{enumerate}
	\end{ti}

	\begin{ti}
		设 $f(x) = g'(x)$,
		\[
			g(x) = \begin{cases}
				\frac{\ee^{x} - 1}{x}, & x \ne 0,\\
				1, & x = 0,
			\end{cases}
		\]
		求 $f^{(n)}(0)$.
	\end{ti}
	\input{chapter/sec2-3.tex}
	\input{chapter/sec2-4.tex}
	\input{chapter/sec3-1.tex}
	\input{chapter/sec3-2.tex}
	\input{chapter/sec3-3.tex}
	\subsection{分部积分法}

	\begin{ti}
		求 $\int_{0}^{1} \ln \bigl( x + \sqrt{x^{2} + 3} \bigr) \dd{x}$.
	\end{ti}

	\begin{ti}
		$\int \frac{x \ln ( x + \sqrt{1 + x^{2}} )}{\left( 1 + x^{2} \right)^{2}} \dd{x} = $\htwo.
	\end{ti}

	\begin{ti}
		$\int \frac{x \ln x}{\left( x^{2} - 1 \right)^{\frac{3}{2}}} \dd{x}$.
	\end{ti}

	\begin{ti}
		求 $\int_{-\frac{\uppi}{4}}^{\frac{\uppi}{4}} 5 \cos x \cdot \arctan \ee^{x} \dd{x}$.
	\end{ti}

	\begin{ti}
		求 $\int x \arctan x \dd{x}$.
	\end{ti}

	\begin{ti}
		求 $\int_{0}^{+\infty} \frac{x \ee^{-3x}}{\left( 1 + \ee^{-3x} \right)^{2}} \dd{x}$.
	\end{ti}

	\begin{ti}
		求 $\int_{0}^{1} x \ln (1 - x) \dd{x}$.
	\end{ti}

	\begin{ti}
		设 $f\bigl( \sin^{2}x \bigr) = \frac{x}{\sin x}$,求 $\int \frac{\sqrt{x}}{\sqrt{1 - x}} f(x) \dd{x}$.
	\end{ti}

	\begin{ti}
		已知 $f(x) = \frac{\ee^{x} + \ee^{-x}}{2}$,求 $\int \Bigl[ \frac{f'(x)}{f(x)} + \frac{f(x)}{f'(x)} \Bigr] \dd{x}$.
	\end{ti}

	\begin{ti}
		已知 $\int_{0}^{1} f(x) \dd{x} = 1$,$f(1) = 0$,则 \[\int_{0}^{1} x f'(x) \dd{x} = \]
		\htwo.
	\end{ti}

	\begin{ti}
		已知 $f(x)$ 的一个原函数为 $(1 + \sin x) \ln x$,求 $\int x f'(x) \dd{x}$.
	\end{ti}

	\begin{ti}
		设 $\frac{\ln x}{x}$ 是 $f(x)$ 的一个原函数,则 $\int_{1}^{\ee} x\times f'(x) \dd{x} = $ \htwo.
	\end{ti}

	\begin{ti}
		设 $f(x)$ 有一个原函数 $\frac{\sin x}{x}$,则 $\int_{\frac{\uppi}{2}}^{\uppi} x^{3} f'(x) \dd{x} = $
		\htwo.
	\end{ti}

	\begin{ti}
		设 $f(x) = \lim_{t \to \infty} t^{2} \sin \frac{x}{t} \cdot \bigl[ g\bigl( 2x + \frac{1}{t} \bigr) - g(2x) \bigr]$,$g(x)$ 的一个原函数为 $\ln(x + 1)$,求 $\int_{0}^{1} f(x) \dd{x}$.
	\end{ti}

	\begin{ti}
		设 $f(x)$ 的一个原函数 \[F(x) = \ln^{2}\bigl( x + \sqrt{1 + x^{2}} \bigr),\]求 $\int x f'(x) \dd{x}$.
	\end{ti}

	\begin{ti}
		求 $I = \int_{0}^{1} \frac{f(x)}{\sqrt{x}} \dd{x}$,其中 $f(x) = \int_{1}^{\sqrt{x}} \ee^{-t^{2}} \dd{t}$.
	\end{ti}

	\begin{ti}
		设 $f(x) = \int_{0}^{x} \ee^{-t^{2} + 2t} \dd{t}$,求 $\int_{0}^{1} (x - 1)^{2} f(x) \dd{x}$.
	\end{ti}

	\begin{ti}
		设 $y'(x) = \arctan (x - 1)^{2}$,且 $y(0) = 0$,求
		\[
			\int_{0}^{1} y(x) \dd{x}.
		\]
	\end{ti}
	\input{chapter/sec3-5.tex}
	\input{chapter/sec3-6.tex}
	\input{chapter/sec3-7.tex}
	\input{chapter/sec3-8.tex}
	\input{chapter/sec3-9.tex}
	\input{chapter/sec3-10.tex}
	\input{chapter/sec3-11.tex}
	\input{chapter/sec3-12.tex}
	\input{chapter/sec3-13.tex}
	\input{chapter/sec3-14.tex}
	\input{chapter/sec3-15.tex}
	\input{chapter/sec4-1.tex}
	\input{chapter/sec4-2.tex}
	\input{chapter/sec4-3.tex}
	\input{chapter/sec5-1.tex}
	\input{chapter/sec5-2.tex}
	\subsection{计算}
	
	\begin{ti}
		计算 $I = \int_{1}^{2} \dd{x} \int_{\frac{1}{x}}^{1} y \ee^{xy} \dd{y}$.
	\end{ti}

	\begin{ti}
		\[
			\int_{0}^{1} \dd{y} \int_{0}^{1} \sqrt{\ee^{2x} - y^{2}} \dd{x} + \int_{1}^{\ee} \dd{y} \int_{\ln y}^{1} \sqrt{\ee^{2x} - y^{2}} \dd{x} = 
		\]
		\kuo.

		\twoch{$\frac{\uppi}{8}\bigl( \ee^{2} - 1 \bigr)$}{$\frac{\uppi}{8}\bigl( \ee^{2} + 1 \bigr)$}{$\frac{\uppi}{4}\bigl( \ee^{2} - 1 \bigr)$}{$\frac{\uppi}{4}\bigl( \ee^{2} + 1 \bigr)$}
	\end{ti}

	\begin{ti}
		已知
		\[
			I = \int_{0}^{2} \dd{x} \int_{0}^{\frac{x^{2}}{2}} f(x,y) \dd{y} + \int_{2}^{2\sqrt{2}} \dd{x} \int_{0}^{\sqrt{8 - x^{2}}} f(x,y) \dd{y},
		\]
		则 $I = $\kuo.
		\onech{$\int_{0}^{2} \dd{y} \int_{\sqrt{2y}}^{\sqrt{8 - y^{2}}} f(x,y) \dd{x}$}{$\int_{0}^{2} \dd{y} \int_{1}^{\sqrt{8 - y^{2}}} f(x,y) \dd{x}$}{$\int_{0}^{1} \dd{y} \int_{\sqrt{2y}}^{\sqrt{8 - y^{2}}} f(x,y) \dd{x}$}{$\int_{0}^{2} \dd{y} \int_{\sqrt{2y}}^{1} f(x,y) \dd{x}$}
	\end{ti}

	\begin{ti}
		累次积分 $\int_{0}^{2R} \dd{y} \int_{0}^{\sqrt{2Ry - y^{2}}} f \bigl( x^{2} + y^{2} \bigr) \dd{x} (R > 0)$ 化为极坐标形式的累次积分为\kuo.

		\onech{$\int_{0}^{\uppi} \dd{\theta} \int_{0}^{2R\sin\theta} f \bigl( r^{2} \bigr) r \dd{r}$}{$\int_{0}^{\frac{\uppi}{2}} \dd{\theta} \int_{0}^{2R\cos\theta} f \bigl( r^{2} \bigr) r \dd{r}$}{$\int_{0}^{\frac{\uppi}{2}} \dd{\theta} \int_{0}^{2R\sin\theta} f \bigl( r^{2} \bigr) r \dd{r}$}{$\int_{0}^{\uppi} \dd{\theta} \int_{0}^{2R\cos\theta} f \bigl( r^{2} \bigr) r \dd{r}$}
	\end{ti}

	\begin{ti}
		计算 $\int_{0}^{1} \dd{y} \int_{\arcsin y}^{\frac{\uppi}{2}} \cos x \cdot \sqrt{1 + \cos^{2}x} \dd{x}$.
	\end{ti}

	\begin{ti}
		计算 $\int_{0}^{1} \dd{y} \int_{3y}^{3} \ee^{x^{2}} \dd{x}$.
	\end{ti}

	\begin{ti}
		计算 $\int_{0}^{1} \dd{y} \int_{\sqrt{y}}^{1} \sqrt{x^{3} + 1} \dd{x}$.
	\end{ti}

	\begin{ti}
		计算 $\int_{0}^{1} \dd{x} \int_{x^{2}}^{1} x^{3} \sin y^{3} \dd{y}$.
	\end{ti}

	\begin{ti}
		计算 $\int_{0}^{1} \dd{x} \int_{x^{2}}^{x} \bigl( x^{2} + y^{2} \bigr)^{-\frac{1}{2}} \dd{y}$.
	\end{ti}

	\begin{ti}
		计算 $\int_{1}^{2} \dd{x} \int_{0}^{x} \frac{y \sqrt{x^{2} + y^{2}}}{x} \dd{y}$.
	\end{ti}

	\begin{ti}
		计算 $\int_{1}^{2} \dd{x} \int_{\sqrt{x}}^{x} \sin \frac{\uppi x}{2y} \dd{y} + \int_{2}^{4} \dd{x} \int_{\sqrt{x}}^{2} \sin \frac{\uppi x}{2y} \dd{y}$.
	\end{ti}

	\begin{ti}
		$\int_{0}^{1} \dd{y} \int_{y}^{1} \Bigl( \frac{\ee^{x^{2}}}{x} - \ee^{y^{2}} \Bigr) \dd{x} = $\htwo.
	\end{ti}

	\begin{ti}
		计算 $\iint_{D} \ee^{\frac{y}{x + y}} \dd{\sigma}$,其中 $D = \bigl\{ (x,y) \bigl| 0 \leq y \leq 1 - x, y \leq x \bigr\}$.
	\end{ti}

	\begin{ti}
		设平面区域 $D = \Bigl\{ (x,y) \Bigl| x^{2} + y^{2} \leq 8, y \geq \frac{x^{2}}{2} \Bigr\}$,计算
		\[
			I = \iint_{D} \bigl[ (x - 1)^{2} + y^{2} \bigr] \dd{\sigma}.
		\]
	\end{ti}

	\begin{ti}
		计算
		\[
			I = \iint_{D} \bigl( x^{2} + xy \bigr)^{2} \dd{x} \dd{y},
		\]
		其中 $D = \bigl\{ (x,y) \bigl| x^{2} + y^{2} \leq 2x \bigr\}$.
	\end{ti}

	\begin{ti}
		计算 $I = \iint_{\sqrt{x} + \sqrt{y} \leq 1} \sqrt[3]{\sqrt{x} + \sqrt{y}} \dd{x} \dd{y}$.
	\end{ti}

	\begin{ti}
		设函数 $f(x,y)$ 连续,且
		\[
			f(x,y) = x + \iint_{D} y f(u,v) \dd{u} \dd{v},
		\]
		其中 $D$ 由 $y = \frac{1}{x}$,$x = 1$,$y = 2$ 围成,求 $f(x,y)$.
	\end{ti}

	\begin{ti}
		设 $D = \bigl\{ (x,y) \bigl| |x| \leq 2, |y| \leq 2 \bigr\}$,计算
		\[
			I = \iint_{D} \bigl| x^{2} + y^{2} - 1 \bigr| \dd{\sigma}.
		\]
	\end{ti}

	\begin{ti}
		设 $D = \bigl\{ (x,y) \bigl| 0 \leq x \leq 1, 0 \leq y \leq 2\ee \bigr\}$,计算
		\[
			\iint_{D} x \bigl| y - \ee^{x} \bigr| \dd{\sigma}.
		\]
	\end{ti}

	\begin{ti}
		计算 $I = \iint_{D} \bigl( |x| + |y| \bigr) \dd{x} \dd{y}$,其中 $D$ 是由曲线 $xy = 2$,直线 $y = x - 1$ 及 $y = x + 1$ 所围成的区域.
	\end{ti}

	\begin{ti}
		设 $D = \bigl\{ (x,y) \bigl| 0 \leq x \leq \uppi, 0 \leq y \leq 2 \bigr\}$,计算 $\iint_{D} \bigl| y - \sin x \bigr| \dd{\sigma}$.
	\end{ti}

	\begin{ti}
		计算
		\[
			I = \int_{-1}^{1} \dd{x} \int_{x}^{2 - |x|} \bigl[ \ee^{|y|} + \sin \bigl( x^{3}y^{3} \bigr) \bigr] \dd{y}.
		\]
	\end{ti}

	\begin{ti}
		设 $f(x,y) = \begin{cases}
			1 - x - y, & x + y \leq 1,\\
			2, & x + y > 1,
		\end{cases}$ 计算
		\[
			\iint_{D} f(x,y) \dd{x} \dd{y},
		\]
		其中 $D$ 为正方形区域 $\bigl\{ (x,y) \bigl| 0 \leq x \leq 1, 0 \leq y \leq 1 \bigr\}$.
	\end{ti}

	\begin{ti}
		设函数 $f(x) = \begin{cases}
			x, & 0 \leq x \leq 2,\\
			0, & x < 0 \text{\ 或\ } x > 2,
		\end{cases}$ 计算 $I = \iint_{D} \frac{f(x + y)}{f\left( \sqrt{x^{2} + y^{2}} \right)} \dd{x} \dd{y}$,其中 $D = \bigl\{ (x,y) \bigl| x^{2} + y^{2} \leq 4 \bigr\}$.
	\end{ti}

	\begin{ti}
		计算 $\iint_{D} \min\bigl\{ x,y \bigr\} \dd{x} \dd{y}$,其中 $D = \bigl\{ (x,y) \bigl| 0 \leq x \leq 3, 0 \leq y \leq 1 \bigr\}$.
	\end{ti}

	\begin{ti}
		计算 $\int_{0}^{a} \dd{x} \int_{0}^{b} \ee^{ \max\left\{ b^{2}x^{2}, a^{2}y^{2} \right\} } \dd{y}$,其中 $a,b > 0$.
	\end{ti}

	\begin{ti}
		设 $F(x,y) = \frac{\partial^{2}f(x,y)}{\partial x \partial y}$ 在 $D = [a,b] \times [c,d]$ 上连续,求
		\[
			I = \iint_{D} F(x,y) \dd{x} \dd{y},
		\]
		并证明:$I \leq 2(M - m)$,其中 $M$ 和 $m$ 分别是 $f(x,y)$ 在 $D$ 上的最大值和最小值.
	\end{ti}

	\begin{ti}
		设函数 $f(x)$ 在 $[0,1]$ 上连续,证明:
		\[
			\int_{0}^{1} \ee^{f(x)} \dd{x} \int_{0}^{1} \ee^{-f(y)} \dd{y} \geq 1.
		\]
	\end{ti}

	\begin{ti}
		设 $f(x,y)$ 为连续函数,则
		\[
			I = \lim_{t \to 0^{+}} \frac{1}{\uppi t^{2}} \iint_{D} f(x,y) \dd{\sigma}
		\] = \htwo,其中 $D = \bigl\{ (x,y) \bigl| x^{2} + y^{2} \leq t^{2} \bigr\}$.
	\end{ti}

	\begin{ti}
		已知 $f(t) = \iint_{D(t): x^{2} + y^{2} \leq t^{2}} \bigl( \ee^{x^{2} + y^{2}} - ky^{2} \bigr) \dd{\sigma}$ 在 $t \in (0,+\infty)$ 内是单调增加函数,$k$为常数,求 $k$ 的最大取值范围.
	\end{ti}

	\begin{ti}
		由曲线 $y = x^{2}$,$y = x + 2$ 所围成的平面薄片,其上各点处的面密度 $\mu = 1 + x^{2}$,则此薄片的质量 $M = $\htwo.
	\end{ti}

	\begin{ti}
		求柱体 $x^{2} + y^{2} \leq 2x$ 被 $x^{2} + y^{2} + z^{2} = 4$ 所截得部分的体积.
	\end{ti}

	\begin{ti}
		设平面薄片所占的区域 $D$ 由抛物线 $y = x^{2}$ 及直线 $y = x$ 所围成,它在 $(x,y)$ 处的面密度 $\rho(x,y) = x^{2}y$,求此薄片的重心.
	\end{ti}
	\section{代数与几何}

	\begin{ti}
		已知曲面 $z = x^{2} + y^{2}$ 上点 $P$ 处的切平面平行于平面 $2x + 2y + z - 1 = 0$,则点 $P$ 的坐标是\kuo.

		\twoch{$(1,-1,2)$}{$(-1,1,2)$}{$(1,1,2)$}{$(-1,-1,2)$}
	\end{ti}

	\begin{ti}
		过点 $P(2,0,3)$ 且与直线 \[\begin{cases}
			x - 2y + 4z - 7 = 0,\\
			3x + 5y - 2z + 1 = 0
		\end{cases}\] 垂直的平面的方程是\kuo.

		\onech{$(x-2) - 2(y-0) + 4(z-3) = 0$}{$3(x-2) + 5(y-0) - 2(z-3) = 0$}{$-16(x-2) + 14(y-0) + 11(z-3) = 0$}{$-16(x+2) + 14(y-0) + 11(z-3) = 0$}
	\end{ti}

	\begin{ti}
		已知 $|\bm a| = 1$,$|\bm b| = \sqrt{2}$,且 $( \widehat{ \bm a, \bm b } ) = \frac{\uppi}{4}$,则 $|\bm a + \bm b| = $\kuo.

		\fourch{$1$}{$1 + \sqrt{2}$}{$2$}{$\sqrt{5}$}
	\end{ti}

	\begin{ti}
		曲线 $x^{2} + y^{2} + z^{2} = a^{2}$ 与 $x^{2} + y^{2} = 2az(a > 0)$ 的交线是\kuo.

		\fourch{抛物线}{双曲线}{圆}{椭圆}
	\end{ti}

	\begin{ti}
		若非零向量 $\bm a, \bm b$ 满足关系式 $|\bm a - \bm b| = |\bm a + \bm b|$,则必有\kuo.

		\twoch{$\bm a - \bm b = \bm a + \bm b$}{$\bm a = \bm b$}{$\bm a \cdot \bm b = 0$}{$\bm a \times \bm b = \bm 0$}
	\end{ti}

	\begin{ti}
		若 $\bm a \perp \bm b$,$\bm a, \bm b$ 均为非零向量,$x$ 是非零实数,则有\kuo.

		\twoch{$|\bm a + x \bm b| > |\bm a| + |x| |\bm b|$}{$|\bm a - x \bm b| < |\bm a| - |x| |\bm b|$}{$|\bm a + x \bm b| > |\bm a|$}{$|\bm a - x \bm b| < |\bm a|$}
	\end{ti}

	\begin{ti}
		在曲线 $x = t, y = -t^{2}, z = t^{3}$ 的所有切线中,与平面 $x + 2y + z = 4$ 平行的切线\kuo.

		\twoch{只有 $1$ 条}{只有 $2$ 条}{至少有 $3$ 条}{不存在}
	\end{ti}

	\begin{ti}
		两条平行直线
		\begin{align*}
			L_{1}&: \begin{cases}
				x = 1 + t,\\
				y = -1 + 2t,\\
				z = t,
			\end{cases},\\
			L_{2}&: \begin{cases}
				x = 2 + t,\\
				y = -1 + 2t,\\
				z = 1 + t
			\end{cases},
		\end{align*}
		之间的距离为\kuo.

		\fourch{$\frac{2}{3}$}{$\frac{2}{3}\sqrt{3}$}{$1$}{$2$}
	\end{ti}

	\begin{ti}
		与直线 $L_{1}: \begin{cases}
			x = 1,\\
			y = -2 + t,\\
			z = 1 + t
		\end{cases}$ 及直线
		\[
			L_{2}: \frac{x + 1}{1} = \frac{y + 1}{2} = \frac{z - 1}{1}
		\]
		都平行,且过原点的平面 $\pi$ 的方程为\kuo.

		\twoch{$x + y + z = 0$}{$x - y + z = 0$}{$x + y - z = 0$}{$x - y + z + 2 = 0$}
	\end{ti}

	\begin{ti}
		直线 $L: \frac{x - 2}{2} = \frac{y - 1}{1} = \frac{z - 3}{1}$ 与平面 $\pi: x - y + 2z + 4 = 0$ 的夹角为\kuo.

		\fourch{$\pi$}{$\frac{\pi}{3}$}{$\frac{\pi}{6}$}{$\frac{\pi}{2}$}
	\end{ti}

	\begin{ti}
		已知等边三角形 $\triangle ABC$ 的边长为 $1$,且 $\overrightarrow{BC} = \bm a$,$\overrightarrow{CA} = \bm b$,$\overrightarrow{AB} = \bm c$,则 $\bm a \cdot \bm b + \bm b \cdot \bm c + \bm c \cdot \bm a = $\kuo.

		\fourch{$\frac{1}{2}$}{$\frac{2}{3}$}{$-\frac{1}{2}$}{$-\frac{3}{2}$}
	\end{ti}

	\begin{ti}
		设直线 $L$ 为 $\begin{cases}
			x + 3y + 2z + 1 = 0,\\
			2x - y - 10z + 3 = 0,
		\end{cases}$ 平面 $\pi$ 为 $4x - 2y + z - 2 = 0$,则\kuo.

		\twoch{$L$ 平行于 $\pi$}{$L$ 在 $\pi$ 上}{$L$ 垂直于 $\pi$}{$L$ 与 $\pi$ 相交但不垂直}
	\end{ti}

	\begin{ti}
		设 $\bm a$ 与 $\bm b$ 为非零向量,则 $\bm a \times \bm b = \bm 0$ 是\kuo.

		\onech{$\bm a = \bm b$ 的充要条件}{$\bm a \perp \bm b$ 的充要条件}{$\bm a \parallel \bm b$ 的充要条件}{$\bm a \parallel \bm b$ 的必要但不充分条件}
	\end{ti}

	\begin{ti}
		设 $\bm c = \alpha \bm a + \beta \bm b$,$\bm a, \bm b$ 为非零向量,且 $\bm a$ 与 $\bm b$ 不平行. 若这些向量起点相同,且 $\bm a, \bm b, \bm c$ 的终点在同一直线上,则必有\kuo.

		\twoch{$\alpha \beta \geq 0$}{$\alpha \beta \leq 0$}{$\alpha + \beta = 1$}{$\alpha^{2} + \beta^{2} = 1$}
	\end{ti}

	\begin{ti}
		设有直线
		\[
			L_{1}: \frac{x - 1}{1} = \frac{y - 5}{-2} = \frac{z + 8}{1}
		\]
		与 $L_{2}: \begin{cases}
			x - y = 6,\\
			2y + z = 3,
		\end{cases}$ 则 $L_{1}$ 与 $L_{2}$ 的夹角为\kuo.

		\fourch{$\frac{\pi}{3}$}{$\frac{\pi}{6}$}{$\frac{\pi}{4}$}{$\frac{\pi}{2}$}
	\end{ti}

	\begin{ti}
		曲线 $S: \begin{cases}
			x^{2} + y^{2} + z^{2} = 2,\\
			x + y + z = 0
		\end{cases}$ 在点 $(1,-1,0)$ 处的切线方程为\kuo.

		\twoch{$\frac{x - 1}{2} = \frac{y + 1}{1} = \frac{z}{1}$}{$\frac{x - 1}{2} = \frac{y + 1}{2} = \frac{z}{3}$}{$\frac{x - 1}{-1} = \frac{y + 1}{-1} = \frac{z}{1}$}{$\frac{x - 1}{1} = \frac{y + 1}{1} = \frac{z}{-2}$}
	\end{ti}

	\begin{ti}
		曲面 $x^{\frac{2}{3}} + y^{\frac{2}{3}} + z^{\frac{2}{3}} = 4$ 上任一点的切平面在三个坐标轴上的截距的平方和为\kuo.

		\fourch{$48$}{$64$}{$36$}{$16$}
	\end{ti}

	\begin{ti}
		以下 $4$ 个平面方程:\libcirc{1}$7x + 5y + 2z + 10 = 0$,\libcirc{2}$-7x - 5y + 2z - 10 = 0$,\libcirc{3}$7x - y + 14z + 26 = 0$,\libcirc{4}$x - 7y + 14z - 26 = 0$,是平面 $x + 2y - 2z + 6 = 0$ 和平面 $4x - y + 8z - 8 = 0$ 的交角的平分面方程的是\kuo.

		\fourch{\circled{1}\circled{2}}{\circled{2}\circled{3}}{\circled{2}\circled{4}}{\circled{1}\circled{4}}
	\end{ti}

	\begin{ti}
		已知 $|\bm a| = 2$,$|\bm b| = 2$,$(\widehat{\bm a, \bm b}) = \frac{\uppi}{3}$,则 $\bm u = 2 \bm a - 3 \bm b$ 的模 $|\bm u| = $\htwo.
	\end{ti}

	\begin{ti}
		设 $\bm A = 2 \bm a + \bm b$,$\bm B = k \bm a + \bm b$,其中 $|\bm a| = 1$,$|\bm b| = 2$,且 $\bm a \perp \bm b$. 若 $\bm A \perp \bm B$,则 $k = $\htwo.
	\end{ti}

	\begin{ti}
		点 $(-1,2,0)$ 在平面 $x + 2y - z + 1 = 0$ 上的投影为\htwo.
	\end{ti}

	\begin{ti}
		点 $(1,2,1)$ 到平面 $x + 2y + 2z - 13 = 0$ 的距离是 \htwo.
	\end{ti}

	\begin{ti}
		过三点 $A(1,1,-1), B(-2,-2,2)$ 和 $C(1,-1,2)$ 的平面方程是\htwo.
	\end{ti}

	\begin{ti}
		$xOz$ 坐标面上的抛物线 $z^{2} = x - 2$ 绕 $x$ 轴旋转而成的旋转抛物面的方程是\htwo.
	\end{ti}

	\begin{ti}
		设 $\bm a, \bm b, \bm c$ 的模 $|\bm a| = |\bm b| = |\bm c| = 2$,且满足 $\bm a + \bm b + \bm c = \bm 0$,则 $\bm a \cdot \bm b + \bm b \cdot \bm c + \bm c \cdot \bm a = $\htwo.
	\end{ti}

	\begin{ti}
		经过点 $A(1,0,0)$ 与点 $B(0,1,1)$ 的直线绕 $z$ 轴旋转一周生成的曲面方程是\htwo.
	\end{ti}

	\begin{ti}
		已知直线
		\[
			L_{1}: \begin{cases}
				x + y = 0,\\
				2y + z + 1 = 0
			\end{cases}
		\]
		和
		\[
			L_{2}: \begin{cases}
				x = 1 - t,\\
				y = -1 + 2t,\\
				z = 1 + t,
			\end{cases}
		\]
		则过直线 $L_{1}$ 和 $L_{2}$ 的平面是\htwo.
	\end{ti}

	\begin{ti}
		过直线 $\begin{cases}
			x = 1 + t,\\
			y = 1 + 2t,\\
			z = 1 + 3t
		\end{cases}$ 且和点 $(2,2,2)$ 的距离为 $\frac{1}{\sqrt{3}}$ 的平面方程是\htwo.
	\end{ti}

	\begin{ti}
		曲面 $z - \ee^{z} + 2xy = 3$ 在点 $(1,2,0)$ 处的切平面方程为\htwo.
	\end{ti}

	\begin{ti}
		点 $(1,2,3)$ 到直线 $\frac{x}{1} = \frac{y - 4}{-3} = \frac{z - 3}{-2}$ 的距离为\htwo.
	\end{ti}

	\begin{ti}
		设 $\bm a = 3 \bm i + 4 \bm k$,$\bm b = - \bm i + 2 \bm j - 2 \bm k$,求与向量 $\bm a$ 和 $\bm b$ 均垂直的单位向量.
	\end{ti}

	\begin{ti}
		求到平面 $2x - 3y + 6z - 4 = 0$ 和平面 $12x - 15y + 16z - 1 = 0$ 距离相等的点的轨迹方程.
	\end{ti}

	\begin{ti}
		求过两点 $A(0,1,0), B(-1,2,1)$ 且与直线
		\[
			\begin{cases}
				x = -2 + t,\\
				y = 1 - 4t,\\
				z = 2 + 3t
			\end{cases}
		\]
		平行的平面方程.
	\end{ti}

	\begin{ti}
		求直线 $L: \frac{x - 3}{2} = \frac{y - 1}{3} = z + 1$ 绕直线
		\[
			L_{1}: \begin{cases}
				x = 2,\\
				y = 3
			\end{cases}
		\]
		旋转一周所成的曲面方程.
	\end{ti}

	\begin{ti}
		确定下列直线与平面的位置关系(垂直、平行、在平面上):
		\begin{enumerate}
			\item $L: \begin{cases}
				x - y + 2z - 3 = 0,\\
				x = y,
			\end{cases} \pi: x + y - 6 = 0$;
			\item $L: \begin{cases}
				x + 2y - 3z -4 = 0,\\
				-2x + 6y - 3 = 0,
			\end{cases} \pi: 2x - y - 3z + 7 = 0$;
			\item $L: \frac{x - 1}{-1} = \frac{y - 1}{0} = \frac{z + 2}{2}, \pi: 2x - y + z + 1 = 0$.
		\end{enumerate}
	\end{ti}

	\begin{ti}
		求经过直线 $L: \frac{x - 6}{2} = \frac{y - 3}{1} = \frac{2z - 1}{-2}$ 且与椭球面 $S: x^{2} + 2y^{2} + 3z^{2} = 21$ 相切的切平面方程.
	\end{ti}

	\begin{ti}
		设有空间直线 $l: \frac{x - 1}{1} = \frac{y}{1} = \frac{z - 1}{-1}$ 和平面 $\pi: x - y + 2z - 1 = 0$,求
		\begin{enumerate}
			\item 直线 $l$ 在平面 $\pi$ 上的投影直线 $l_{0}$ 的方程;
			\item 投影直线 $l_{0}$ 绕 $y$ 轴旋转一周所成的旋转曲面方程 $F(x,y,z) = 0$.
		\end{enumerate}
	\end{ti}

	\begin{ti}
		设有直线
		\[
			L: \begin{cases}
				2x + y = 0,\\
				4x + 2y + 3z = 6
			\end{cases}
		\]
		和曲线
		\[
			C: \begin{cases}
				x^{2} + y^{2} + z^{2} = 6,\\
				x + y + z = 0.
			\end{cases}
		\]
		\begin{enumerate}
			\item 求曲线 $C$ 在点 $(1,-2,1)$ 处的切线和法平面方程;
			\item 求通过直线 $L$ 且与平面 $x - z = 0$ 垂直的平面方程.
		\end{enumerate}
	\end{ti}

	\begin{ti}
		求过点 $(1,2,3)$ 且与曲面 $z = x + (y - z)^{3}$ 的所有切平面皆垂直的平面方程.
	\end{ti}
	\input{chapter/sec7-1.tex}
	\input{chapter/sec7-2.tex}
	\subsection{第一型曲面积分}

	\begin{ti}
		设 $\varSigma$ 是 $yOz$ 平面上的圆域 $y^{2} + z^{2} \leq 1$,则 $\iint_{\varSigma} \bigl( x^{2} + y^{2} + z^{2} \bigr) \dd{S} = $\kuo.

		\fourch{$0$}{$\uppi$}{$\frac{1}{4}\uppi$}{$\frac{1}{2}\uppi$}
	\end{ti}

	\begin{ti}
		设 $\varSigma$ 为球面 $(x - 1)^{2} + y^{2} + (z + 1)^{2} = 1$,则 $\iint_{\varSigma} (2x + 3y + z) \dd{S} = $\kuo.
		
		\fourch{$4\uppi$}{$2\uppi$}{$\uppi$}{$0$}
	\end{ti}

	\begin{ti}
		设 $S$ 为椭球面 $\frac{x^{2}}{9} + \frac{y^{2}}{4} + z^{2} = 1$,已知 $S$ 的面积为 $A$,则第一型曲面积分 $\iint_{S} \bigl[ (2x + 3y)^{2} + (6z - 1)^{2} \bigr] \dd{S} = $ \htwo.
	\end{ti}

	\begin{ti}
		设 $S$ 为球面 $x^{2} + y^{2} + z^{2} = R^{2}$ 被锥面 $z = \sqrt{Ax^{2} + By^{2}}$ 截下的小的那部分,并设其中 $A, B, R$ 均为正常数且 $A \ne B$,则第一型曲面积分 $\iint_{S} z \dd{S} = $ \htwo.
	\end{ti}

	\begin{ti}
		设 $\varSigma$ 是正圆锥面 $z = \sqrt{x^{2} + y^{2}} (0 \leq z \leq 1)$,则曲面积分 $\iint_{\varSigma} z \dd{S} = $\kuo.

		\fourch{$\frac{2\sqrt{2}}{3}\uppi$}{$\frac{\sqrt{2}}{3}\uppi$}{$\sqrt{2}\uppi$}{$\uppi$}
	\end{ti}

	\begin{ti}
		空间曲面 $z = xy$ 被圆柱体 $x^{2} + y^{2} \leq 1$ 所截部分的面积 $A = $\htwo.
	\end{ti}

	\begin{ti}
		设 $\varSigma$ 为球面:$x^{2} + y^{2} + z^{2} = 1$,则第一类曲面积分 $\iint_{\varSigma} x (4x - z) \dd{S} = $\htwo.
	\end{ti}

	\begin{ti}
		计算曲面积分 $I = \iint_{\varSigma} (ax + by + cz + d)^{2} \dd{S}$,其中 $\varSigma$ 是球面 $x^{2} + y^{2} + z^{2} = R^{2}$.
	\end{ti}

	\begin{ti}
		设 $\varSigma$ 为平面 $y + z = 5$ 被柱面 $x^{2} + y^{2} = 25$ 所截得的部分,计算曲面积分 $I = \iint_{\varSigma} (x + y + z) \dd{S}$.
	\end{ti}

	\begin{ti}
		计算 $\iint_{S} x^{2} \dd{S}$,其中 $S$ 为圆柱面 $x^{2} + y^{2} = a^{2}$ 介于 $z = 0$ 和 $z = h$ 之间的部分.
	\end{ti}

	\begin{ti}
		设半径为 $R$ 的球的球心位于以原点为中心、$a$ 为半径的定球面上($2a > R > 0$,$a$ 为常数). 试确定 $R$ 为何值时前者夹在定球面内部的表面积为最大,并求出此最大值.
	\end{ti}
	\subsection{第二型曲线积分}

	\begin{ti}
		设 $l$ 为自点 $O(0,0)$ 沿曲线 $y = \sin x$ 至点 $A(\uppi,0)$ 的有向弧段,求平面第二型曲线积分
		\[
			I = \int_{l} \bigl[ \ee^{x} \cos y + 2 (x + y) \bigr] \dd{x} + \Biggl( -\ee^{x} \sin y + \frac{3}{2}x \Biggr) \dd{y}.
		\]
	\end{ti}

	\begin{ti}
		设 $L$ 为圆周 $x^{2} + y^{2} = 4$ 正向一周,求
		\[
			I = \oint_{L} y^{3} \dd{x} + \bigl| 3y - x^{2} \bigr| \dd{y}.
		\]
	\end{ti}

	\begin{ti}
		计算曲线积分 $\oint_{L} \frac{y \dd{x} - x \dd{y}}{2\left(x^{2} + y^{2}\right)}$,其中 $L: (x - 1)^{2} + y^{2} = 2$,其方向为逆时针方向.
	\end{ti}

	\begin{ti}
		计算曲线积分 $\int_{C} \sqrt{x^{2} + y^{2}} \dd{x} + \bigl[ 2x + y\ln\bigl( x + \sqrt{x^{2} + y^{2}} \bigr) \bigr] \dd{y}$,其中有向曲线 $C: y = \sqrt{1 - \frac{(x - 3)^{2}}{4}}$,方向从点 $(5,0)$ 到点 $(1,0)$.
	\end{ti}

	\begin{ti}
		计算曲线积分
		\[
			\int_{L} \Biggl( \frac{xy^{2}}{\sqrt{4 + x^{2}y^{2}}} + \frac{1}{\uppi}x \Biggr)\dd{x} + \Biggl( \frac{x^{2}y}{\sqrt{4 + x^{2}y^{2}}} - x + y \Biggr)\dd{y},
		\]
		其中 $L$ 是摆线 $\begin{cases}
			x = a (t - \sin t),\\
			y = a (1 - \cos t)
		\end{cases} (a > 0)$ 上自 $O(0,0)$ 至 $A(2\uppi a,0)$ 的一段有向曲线弧.
	\end{ti}

	\begin{ti}
		计算曲线积分
		\[
			I = \int_{l} \bigl[ u_{x}'(x,y) + xy \bigr] \dd{x} + u_{y}'(x,y) \dd{y},
		\]
		其中 $l$ 是从点 $A(0,1)$ 沿曲线 $y = \frac{\sin x}{x}$ 到点 $B(\uppi,0)$ 的曲线段. $u(x,y)$ 在 $xOy$ 平面上具有二阶连续偏导数,且 $u(0,1) = 1$,$u(\uppi,0) = \uppi$.
	\end{ti}

	\begin{ti}
		设 $y' = f(x,y)$ 是一条简单封闭曲线 $L$(取正向),$f(x,y) \ne 0$,其所围区域记为 $D$,$D$ 的面积为 $1$,则 $I = \oint_{L} xf(x,y) \dd{x} - \frac{y}{f(x,y)} \dd{y} = $\htwo.
	\end{ti}

	\begin{ti}
		在过点 $O(0,0)$ 和 $A(\uppi,0)$ 的曲线族 $y = a \sin x$ $(a > 0)$ 中,求一条曲线 $L$,使沿该曲线从 $O$ 到 $A$ 的积分 $\int_{L} \bigl( 1 + y^{3} \bigr) \dd{x} + (2x + y) \dd{y}$ 的值最小.
	\end{ti}

	\begin{ti}
		证明
		\[
			\oint_{\varGamma} x f(y) \dd{y} - \frac{y}{f(x)} \dd{x} \geq 2\uppi,
		\]
		其中 $\varGamma$ 为圆周曲线 $(x - a)^{2} + (y - a)^{2} = 1 (a > 0)$ 正向,$f(x)$ 连续取正值.
	\end{ti}

	\begin{ti}
		设曲线积分 $\int_{L} \bigl[ f(x) - \ee^{x} \bigr] \sin y \dd{x} - f(x)\times \cos y \dd{y}$ 与路径无关,其中 $f(x)$ 具有一阶连续导数,且 $f(0) = 0$,则 $f(x)$ 等于\kuo.

		\twoch{$\frac{1}{2}\bigl( \ee^{-x} - \ee^{x} \bigr)$}{$\frac{1}{2}\bigl( \ee^{x} - \ee^{-x} \bigr)$}{$\frac{1}{2}\bigl( \ee^{x} + \ee^{-x} \bigr) - 1$}{$1 - \frac{1}{2}\bigl( \ee^{x} + \ee^{-x} \bigr)$}
	\end{ti}

	\begin{ti}
		已知曲线积分 $\int_{L} \bigl[ \ee^{x} \cos y + y f(x) \bigr] \dd{x} + \bigl( x^{3} - \ee^{x} \sin y \bigr) \dd{y}$ 与路径无关且 $f(x)$ 有连续的导数,则 $f(x) = $\htwo.
	\end{ti}

	\begin{ti}
		设 $f(x,y)$ 在全平面有连续偏导数,曲线积分 $\int_{L} f(x,y) \dd{x} + x \cos y \dd{y}$ 在全平面与路径无关,且 $\int_{(0,0)}^{\left(t,t^{2}\right)} f(x,y) \dd{x} + x \cos y \dd{y} = t^{2}$,求 $f(x,y)$.
	\end{ti}

	\begin{ti}
		设函数 $P(x,y) = \frac{x}{y}r^{\lambda}$,$Q(x,y) = - \frac{x^{2}}{y^{2}}r^{\lambda}$,其中 $r = \sqrt{x^{2} + y^{2}}$,若曲线积分 $\int_{L} P \dd{x} + Q \dd{y}$ 在区域 $D = \bigl\{ (x,y) \bigl| y > 0 \bigr\}$ 上与路径无关,求参数 $\lambda$.
	\end{ti}

	\begin{ti}
		设 $L$ 是摆线 $\begin{cases}
			x = t - \sin t - \uppi,\\
			y = 1 - \cos t
		\end{cases}$ 从 $t = 0$ 到 $t = 2\uppi$ 的一段,则 $\int_{L} \frac{(x - y)\dd{x} + (x + y) \dd{y}}{x^{2} + y^{2}} = $\kuo.
		
		\fourch{$-\uppi$}{$\uppi$}{$2\uppi$}{$-2\uppi$}
	\end{ti}

	\begin{ti}
		设函数 $g(x)$ 具有连续导数,曲线积分
		\[
			\int_{L} \bigl[ \ee^{2x} + g'(x) - 2g(x) \bigr]y \dd{x} - g'(x) \dd{y}
		\]
		与路径无关.
		\begin{enumerate}
			\item 求满足条件 $g(0) = -\frac{1}{4}, g'(0) = -\frac{1}{2}$ 的函数 $g(x)$;
			\item 计算 $\int_{(0,0)}^{(1,1)} \bigl[ \ee^{2x} + g'(x) - 2g(x) \bigr]y \dd{x} - g'(x) \dd{y}$ 的值.
		\end{enumerate}
	\end{ti}

	\begin{ti}
		微分方程 $\bigl( 2xy + \ee^{x}\sin y \bigr) \dd{x} + \bigl( x^{2} + \ee^{x}\times \cos y \bigr) \dd{y} = 0$ 的通解为\htwo.
	\end{ti}

	\begin{ti}
		\begin{enumerate}
			\item 设函数 $f(x)$ 具有一阶连续导数,且 $f(1) = 1$,$D$ 为不包含原点的单连通区域,在 $D$ 内曲线积分 $\int_{L} \frac{y\dd{x} - x\dd{y}}{2x^{2} + f(y)}$ 与路径无关,求 $f(y)$;\label{7.46:1}
			\item 在(\ref{7.46:1})的条件下,求 $\oint_{L'} \frac{y\dd{x} - x\dd{y}}{2x^{2} + f(y)}$,其中 $L'$ 为曲线 $x^{\frac{2}{3}} + y^{\frac{2}{3}} = a^{\frac{2}{3}}, a > 0$,且取逆时针方向.
		\end{enumerate}
	\end{ti}

	\begin{ti}
		设 $L$ 为曲线 $x^{2} + y^{2} = R^{2}$(常数 $R > 0$)一周,$\bm n$ 为 $L$ 的外法线方向向量,$u(x,y)$ 具有二阶连续偏导数且 $\frac{\partial^{2}u}{\partial x^{2}} + \frac{\partial^{2}u}{\partial y^{2}} = x^{2} + y^{2}$. 求 $\oint_{L} \frac{\partial u}{\partial \bm n} \dd{s}$.
	\end{ti}

	\begin{ti}
		计算 $\int_{\varGamma} y\dd{x} + z\dd{y} + x\dd{z}$,其中 $\varGamma$ 为螺旋线 $x = a \cos t, y = a \sin t, z = bt$ 从 $t = 0$ 到 $t = 2\uppi$ 的一段,如图~\ref{fig:1.7.3} 所示.
		\begin{figure}[htbp]
			\centering
			\includegraphics[scale=1]{figure/fig1-7-3.pdf}
			\caption{}\label{fig:1.7.3}
		\end{figure}
	\end{ti}
	\subsection{第二型曲面积分}

	\begin{ti}
		设 $\varSigma$ 是球面 $x^{2} + y^{2} + z^{2} = a^{2} (a > 0)$ 的外侧,则 $\oiint_{\varSigma} xy^{2} \dd{y}\dd{z} + yz^{2} \dd{z}\dd{x} + zx^{2} \dd{x}\dd{y} = $\htwo.
	\end{ti}

	\begin{ti}
		设 $\varSigma: x^{2} + y^{2} + z^{2} = 4 (z \geq 0)$,取上侧,试求曲面积分
		\[
			I = \iint_{\varSigma} \frac{x\dd{y}\dd{z} + y\dd{z}\dd{x} + z\dd{x}\dd{y}}{\sqrt{ x^{2} + (y - 1)^{2} + z^{2} }}.
		\]
	\end{ti}

	\begin{ti}
		设 $f(x,y,z)$ 为连续函数,$S$ 为曲面 $z = \frac{1}{2} \bigl( x^{2} + y^{2} \bigr)$ 介于 $z = 2$ 与 $z = 8$ 之间的上侧部分,求
		\begin{align*}
			&\iint_{S} \bigl[ y f(x,y,z) + x \bigr] \dd{y} \dd{z}\\
			&+ \bigl[ x f(x,y,z) + y \bigr] \dd{z} \dd{x}\\
			&+ \bigl[ 2xy f(x,y,z) + z \bigr] \dd{x} \dd{y}.
		\end{align*}
	\end{ti}

	\begin{ti}
		设 $S$ 为平面 $x - y + z = 1$ 介于三坐标平面间的有限部分,法向量与 $z$ 轴交角为锐角,$f(x,y,z)$ 连续,计算
		\begin{align*}
			\iint_{S} \bigl[ f(x,y,z) + x \bigr] \dd{y} \dd{z} &+ \bigl[ 2f(x,y,z) + y \bigr] \dd{z} \dd{x}\\
			&+ \bigl[ f(x,y,z) + z \bigr] \dd{x} \dd{y}.
		\end{align*}
	\end{ti}

	\begin{ti}
		计算曲面积分
		\begin{align*}
			I = \iint_{\varSigma} \bigl( x^{3} + az^{2} \bigr) \dd{y} \dd{z} &+ \bigl( y^{3} + ax^{2} \bigr) \dd{z} \dd{x}\\
			&+ \bigl( z^{3} + ay^{2} \bigr) \dd{x} \dd{y},
		\end{align*}
		其中 $\varSigma$ 为上半球面 $z = \sqrt{a^{2} - x^{2} - y^{2}}$ 的上侧.
	\end{ti}

	\begin{ti}
		计算
		\[
			I = \oiint_{\varSigma} \frac{2 \dd{y} \dd{z}}{x \cos^{2}x} + \frac{\dd{z} \dd{x}}{\cos^{2}y} - \frac{\dd{x} \dd{y}}{z \cos^{2}z},
		\]
		其中 $\varSigma$ 为球面 $x^{2} + y^{2} + z^{2} = 1$ 的外侧.
	\end{ti}

	\begin{ti}
		设向量场
		\begin{align*}
			\bm F = \Biggl( x^{2} y z^{2}, \frac{1}{z} \arctan \frac{y}{z} - x y^{2} z^{2}, &\frac{1}{y} \arctan \frac{y}{z}\\
			&+ z(1 + xyz) \Biggr).
		\end{align*}
		\begin{enumerate}
			\item 计算 $\div \bm F |_{(1,1,1)}$ 的值;
			\item 设空间区域 $\varOmega$ 由锥面 $y^{2} + z^{2} = x^{2}$ 与球面 $x^{2} + y^{2} + z^{2} = a^{2}, x^{2} + y^{2} + z^{2} = 4 a^{2}$ 所围成 $(x > 0)$,其中 $a$ 为正常数,记 $\varOmega$ 表面的外侧为 $\varSigma$,计算积分
			\begin{align*}
				I
				&= \oiint_{\varSigma} x^{2} y z^{2} \dd{y} \dd{z}\\
				&+ \Biggl( \frac{1}{z} \arctan \frac{y}{z} - x y^{2} z^{2} \Biggr) \dd{z} \dd{x}\\
				&+ \Biggl[ \frac{1}{y} \arctan \frac{y}{z} + z(1 + xyz) \Biggr] \dd{x} \dd{y}.
			\end{align*}
		\end{enumerate}
	\end{ti}

	\begin{ti}
		设函数 $f(x,y,z)$ 在区域 $\varOmega = \bigl\{ (x,y,z) \bigl| x^{2} + y^{2} + z^{2} \leq 1 \bigr\}$ 上具有连续的二阶偏导数,且满足
		\[
			\frac{\partial^{2}f}{\partial x^{2}} + \frac{\partial^{2}f}{\partial y^{2}} + \frac{\partial^{2}f}{\partial z^{2}} = \sqrt{x^{2} + y^{2} + z^{2}},
		\]
		计算
		\[
			I = \iiint_{\varOmega} \Biggl( x \frac{\partial f}{\partial x} + y \frac{\partial f}{\partial y} + z \frac{\partial f}{\partial z} \Biggr) \dd{x} \dd{y} \dd{z}.
		\]
	\end{ti}

	\begin{ti}
		计算曲线积分 $I = \oint_{L} y^{2} \dd{x} + z^{2} \dd{y} + x^{2} \dd{z}$,其中曲线 $L$ 为 $\begin{cases}
			x^{2} + y^{2} + z^{2} = 4,\\
			x^{2} + y^{2} = 2x
		\end{cases} (z \geq 0)$,从 $x$ 轴的正向往负向看去,取逆时针方向.
	\end{ti}
	\input{chapter/sec7-6.tex}
	\input{chapter/sec8.tex}
	\section{级数}
	\subsection{正项级数}

	\begin{ti}
		设 $f(x) = \int_{0}^{\sin x} \sin \bigl( t^{2} \bigr) \dd{t}, g(x) = \sum_{n=1}^{\infty} \frac{x^{2n+1}}{n^{n} + 2}$,则 $x \to 0$ 时,$f(x)$ 是 $g(x)$ 的\kuo 无穷小.

		\twoch{低阶}{同阶非等价}{等价}{高阶}
	\end{ti}

	\begin{ti}
		设正项级数 $\sum_{n=1}^{\infty} a_{n}$ 收敛,正项级数 $\sum_{n=1}^{\infty} b_{n}$ 发散,则
		
		\noindent\circled{1}~$\sum_{n=1}^{\infty} a_{n} b_{n}$ 必收敛;\\
		\circled{2}~$\sum_{n=1}^{\infty} a_{n} b_{n}$ 必发散;\\
		\circled{3}~$\sum_{n=1}^{\infty} a_{n}^{2}$ 必收敛;\\
		\circled{4}~$\sum_{n=1}^{\infty} b_{n}^{2}$ 必发散\\
		中结论正确的有\kuo.

		\fourch{$1$ 个}{$2$ 个}{$3$ 个}{$4$ 个}
	\end{ti}

	\begin{ti}
		当级数 $\sum_{n=1}^{\infty} a_{n}^{2}$,$\sum_{n=1}^{\infty} b_{n}^{2}$ 都收敛时,级数 $\sum_{n=1}^{\infty} a_{n} b_{n}$ \kuo.

		\onech{条件收敛}{绝对收敛}{发散}{可能收敛,也可能发散}
	\end{ti}

	\begin{ti}
		判别下列正项级数的敛散性:
		\begin{enumerate}
			\item $\sum_{n=1}^{\infty} \bigl( \frac{n}{3n + 2} \bigr)^{n}$;
			\item $\sum_{n=1}^{\infty} \int_{0}^{\frac{1}{n}} \frac{\sqrt{x}}{1 + x^{2}} \dd{x}$;
			\item $\sum_{n=1}^{\infty} \bigl( \sqrt[3]{n + 1} - \sqrt[3]{n} \bigr)$.
		\end{enumerate}
	\end{ti}

	\begin{ti}
		判别级数 $\sum_{n=1}^{\infty} \int_{0}^{\frac{\uppi}{n}} \frac{\sin x}{1 + x} \dd{x}$ 的敛散性.
	\end{ti}

	\begin{ti}
		设 $u_{n} = \int_{0}^{1} x (1 - x) \sin^{2n}x \dd{x}$,讨论级数 $\sum_{n=1}^{\infty} u_{n}$ 的敛散性.
	\end{ti}

	\begin{ti}
		设 $0 \leq u_{n} \leq \frac{1}{n}$,则下列级数中一定收敛的是\kuo.

		\twoch{$\sum_{n=1}^{\infty} u_{n}$}{$\sum_{n=1}^{\infty} (-1)^{n} u_{n}$}{$\sum_{n=1}^{\infty} \sqrt{u_{n}}$}{$\sum_{n=1}^{\infty} (-1)^{n} u_{n}^{2}$}
	\end{ti}

	\begin{ti}
		求 $\lim_{n \to \infty} \frac{1}{4^{n}} \bigl( 1 + \frac{1}{n} \bigr)^{n^{2}}$.
	\end{ti}

	\begin{ti}
		设 $\sum_{n=1}^{\infty} u_{n}$ 和 $\sum_{n=1}^{\infty} v_{n}$ 都是正项级数. 试证:
		\begin{enumerate}
			\item 若 $\sum_{n=1}^{\infty} u_{n}$ 收敛,则 $\sum_{n=1}^{\infty} \sqrt{u_{n} u_{n+1}}$ 收敛;
			\item 若 $\sum_{n=1}^{\infty} \sqrt{u_{n} u_{n+1}}$ 收敛,$u_{n}$ 单调减少,则 $\sum_{n=1}^{\infty} u_{n}$ 收敛;
			\item 若 $\sum_{n=1}^{\infty} v_{n}$ 和 $\sum_{n=1}^{\infty} u_{n}$ 都收敛,则 $\sum_{n=1}^{\infty} u_{n} v_{n}$ 收敛;
			\item 若 $\sum_{n=1}^{\infty} u_{n}$ 收敛,则 $\sum_{n=1}^{\infty} \frac{u_{n}}{n}$ 收敛. 
		\end{enumerate}
	\end{ti}

	\begin{ti}
		求 $\lim_{n \to \infty} \frac{n! a^{n}}{n^{n}}$($a$ 为常数,$0 < |a| < \ee$).
	\end{ti}

	\begin{ti}
		判别下列级数的敛散性($k > 1, a > 1$):
		\begin{enumerate}
			\item $\sum_{n=1}^{\infty} \frac{n^{k}}{a^{n}}$;
			\item $\sum_{n=1}^{\infty} \frac{a^{n}}{n!}$;
			\item $\sum_{n=1}^{\infty} \frac{n!}{n^{n}}$.
		\end{enumerate}
	\end{ti}

	\begin{ti}
		常数项级数 $\frac{1}{2} + \frac{1}{10} + \frac{1}{2^{2}} + \frac{1}{10 \times 2} + \cdots + \frac{1}{2^{n}} + \frac{1}{10n} + \cdots$ 的敛散性为\htwo.
	\end{ti}

	\begin{ti}
		\begin{enumerate}
			\item 设 $\sum_{n=1}^{\infty} u_{n}$ 为正项级数,证明:$\sum_{n=1}^{\infty} u_{n}$ 收敛的充要条件是其部分和数列 $\{ s_{n} \}$ 有界;
			\item 设 $\{ x_{n} \}$ 为单调递增的有界正数数列,证明:$\sum_{n=1}^{\infty} \Bigl( 1 - \frac{x_{n}}{x_{n+1}} \Bigr)$ 收敛.
		\end{enumerate}
	\end{ti}

	\begin{ti}
		设数列 $\{ a_{n} \}, \{ b_{n} \}$ 满足
		\[
			\ee^{b_{n}} = \ee^{a_{n}} - a_{n} (n = 1,2,3,\cdots),
		\]
		求证:
		\begin{enumerate}
			\item 若 $a_{n} > 0$,则 $b_{n} > 0$;
			\item 若 $a_{n} > 0(n = 1,2,3,\cdots), \sum_{n=1}^{\infty} a_{n}$ 收敛,则 $\sum_{n=1}^{\infty} \frac{b_{n}}{a_{n}}$ 收敛.
		\end{enumerate}
	\end{ti}
	\input{chapter/sec9-2.tex}
	\input{chapter/sec9-3.tex}
	\input{chapter/sec9-4.tex}
	\input{chapter/sec9-5.tex}
	\subsection{傅氏级数}

	\begin{ti}
		设 $f(x) = x + 1 (0 \leq x \leq 1)$,则它以 $2$ 为周期的余弦级数在 $x = 0$ 处收敛于\kuo.

		\fourch{$1$}{$-1$}{$0$}{$\frac{1}{2}$}
	\end{ti}

	\begin{ti}
		若将 $f(x) = \begin{cases}
			x^{2}, & 0 \leq x \leq 1,\\
			0, & 1 \leq x \leq 2
		\end{cases}$ 在 $[0,2]$ 上展开成正弦级数,则该级数的和函数 $S(x)$ 为\htwo.
	\end{ti}

	\begin{ti}
		设 $f(x) = \begin{cases}
			x^{2}, & - \uppi \leq x < 0,\\
			-5, & 0 \leq x < \uppi,
		\end{cases}$ 则其以 $2 \uppi$ 为周期的傅里叶级数在 $x = \pm \uppi$ 处收敛于\htwo.
	\end{ti}

	\begin{ti}
		设 $f(x) = x^{2} (0 < x < 1)$,而
		\[
			S(x) = \sum_{n=1}^{\infty} b_{n} \sin n \uppi x, x \in (-\infty,+\infty),
		\]
		其中
		\[
			b_{n} = 2 \int_{0}^{1} f(x) \cdot \sin n \uppi x \dd{x}, n = 1,2,3,\cdots.
		\]
		则 $S\big( - \frac{1}{2} \bigr) = $\kuo.

		\fourch{$-\frac{1}{4}$}{$\frac{1}{4}$}{$-\frac{1}{2}$}{$\frac{1}{2}$}
	\end{ti}

	\begin{ti}
		设 $f(x) = \begin{cases}
			-1, & -\uppi < x \leq 0,\\
			1 + x^{2}, & 0 < x \leq \uppi,
		\end{cases}$ 则其以 $2\uppi$ 为周期的傅里叶级数在 $x = \uppi$ 处收敛于\htwo.
	\end{ti}

	\begin{ti}
		设 $f(x) = \uppi x + x^{2}, -\uppi \leq x < \uppi$,且周期为 $T = 2\uppi$. 若 $f(x)$ 在 $[-\uppi,\uppi)$ 上的傅里叶级数为
		\[
			\frac{a_{0}}{2} + \sum_{n=1}^{\infty} (a_{n} \cos nx + b_{n} \sin nx),
		\]
		则 $b_{3} = $\htwo.
	\end{ti}

	\begin{ti}
		函数 $f(x) = \begin{cases}
			-1, & -\uppi \leq x < 0,\\
			1, & 0 \leq x \leq \uppi
		\end{cases}$ 在 $[-\uppi,\uppi]$ 上展开为傅里叶级数 $\frac{a_{0}}{2} + \sum_{n=1}^{\infty} (a_{n} \cos nx + b_{n} \sin nx)$,则 $a_{n} = $\htwo,$b_{n} = $\htwo,和函数 $S(x) = $\htwo.
	\end{ti}

	\begin{ti}
		设 $f(x)$ 在区间 $[-\uppi,\uppi]$ 上连续且满足 $f(x + \uppi) = - f(x)$,则 $f(x)$ 的傅里叶系数 $a_{2n} = $\htwo.
	\end{ti}

	\begin{ti}
		设函数 $f(x)$ 是以 $2 \uppi$ 为周期的周期函数,且 $f(x) = \ee^{\alpha x} (0 \leq x < 2\uppi)$,其中 $\alpha \ne 0$,试将 $f(x)$ 展开成傅里叶级数,并求级数 $\sum_{n=1}^{\infty} \frac{1}{1 + n^{2}}$ 的和.
	\end{ti}

	\begin{ti}
		设
		\[
			f(x) = \begin{cases}
				x + 1, & 0 \leq x \leq \uppi,\\
				0, & - \uppi \leq x < 0,
			\end{cases}
		\]
		$S(x) = \frac{a_{0}}{2} + \sum_{n=1}^{\infty} (a_{n} \cos nx + b_{n} \sin nx)$ 是 $f(x)$ 的以 $2 \uppi$ 为周期的傅里叶级数,则 $\sum_{n = 1}^{\infty} (-1)^{n} a_{n} = $\htwo.
	\end{ti}

	\begin{ti}
		将函数 $f(x) = x^{2} (0 \leq x \leq \uppi)$ 展开成余弦级数,并求 $\sum_{n = 1}^{\infty} \frac{1}{n^{2}}$ 的和.
	\end{ti}
	\guanggao
	\input{chapter/chap2.tex}
	\input{chapter/chap3.tex}
\end{document}