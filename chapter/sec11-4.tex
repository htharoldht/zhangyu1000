\section{数字特征}
	\begin{titwo}
		现有 10 张奖券,其中 8 张为 2 元的,2 张为 5 元的。今从中任取 3 张,则奖金的数学期望为 \kuo.

		\fourch{$6$}{$7.8$}{$9$}{$11.2$}
	\end{titwo}

	\begin{titwo}
		设 $X_{1}$, $X_{2}$, $X_{3}$ 相互独立,且均服从参数为 $\lambda$ 的泊松分布,令 $Y = \frac{1}{3} (X_{1} + X_{2} + X_{3})$,则 $Y^{2}$ 的数学期望为 \kuo.

		\fourch{$\frac{1}{3} \lambda$}{$\lambda^{2}$}{$\frac{1}{3} \lambda + \lambda^{2}$}{$\frac{1}{3} \lambda^{2} + \lambda$}
	\end{titwo}

	\begin{titwo}
		设 $X$ 为连续型随机变量,方差存在,则对任意常数 $C$ 和 $\varepsilon > 0$,必有 \kuo.

		\onech{$P\{ |X - C| \geq \varepsilon \} = E(|X - C|)/\varepsilon$}%
		{$P\{ |X - C| \geq \varepsilon \} \geq E(|X - C|)/\varepsilon$}%
		{$P\{ |X - C| \geq \varepsilon \} \leq E(|X - C|)/\varepsilon$}%
		{$P\{ |X - C| \geq \varepsilon \} \leq DX/\varepsilon^{2}$}
	\end{titwo}

	\begin{titwo}
		一袋中有 6 个正品 4 个次品,按下列方式抽样:每次取 1 个,取后放回,共取 $n(n \leq 10)$ 次,其中次品个数记为 $X$;若一次性取出 $n(n \leq 10)$ 个,其中次品个数记为 $Y$. 则下列正确的是 \kuo.

		\onech{$EX > EY$}{$EX < EY$}{$EX = EY$}{若 $n$ 不同,则 $EX$, $EY$ 大小不同}
	\end{titwo}

	\begin{titwo}
		设随机变量 $(X,Y)$ 的概率密度 $f(x,y)$ 满足 $f(x,$ $y) = f(-x,y)$,且 $\rho_{XY}$ 存在,则 $\rho_{XY} = $ \kuo.

		\fourch{$1$}{$0$}{$-1$}{$-1$ 或 $1$}
	\end{titwo}

	\begin{titwo}
		设随机变量 $(X,Y)$ 服从二维正态分布,其边缘分布为 $X \sim N(1,1)$, $Y \sim N(2,4)$, $X$ 与 $Y$ 的相关系数为 $\rho_{XY} = -\frac{1}{2}$,且概率 $P\{ aX + bY \leq 1 \} = \frac{1}{2}$,则\kuo.

		\twoch{$a = \frac{1}{2}$, $b = - \frac{1}{4}$}{$a = \frac{1}{4}$, $b = - \frac{1}{2}$}{$a = -\frac{1}{4}$, $b = \frac{1}{2}$}{$a = \frac{1}{2}$, $b = \frac{1}{4}$}
	\end{titwo}

	\begin{titwo}
		设 $a$ 为区间 $(0,1)$ 上一个定点,随机变量 $X$ 服从 $(0,1)$ 上的均匀分布. 以 $Y$ 表示点 $X$ 到 $a$ 的距离,当 $X$ 与 $Y$ 不相关时,$a = $ \kuo.

		\fourch{$0.1$}{$0.3$}{$0.5$}{$0.7$}
	\end{titwo}

	\begin{titwo}
		设 $X$ 是随机变量,$EX > 0$ 且 $E \bigl( X^{2} \bigr) = 0.7$, $DX = 0.2$,则以下各式成立的是 \kuo.

		\twoch{$P\bigl\{ - \frac{1}{2} < X < \frac{3}{2} \bigr\} \geq 0.2$}%
		{$P\bigl\{ X > \sqrt{2} \bigr\} \geq 0.6$}%
		{$P\bigl\{ 0 < X < \sqrt{2} \bigr\} \geq 0.6$}%
		{$P\bigl\{0 < X < \sqrt{2}\bigr\} \leq 0.6$}
	\end{titwo}
	
	\begin{titwo}
		设随机变量 $X_{1}$, $X_{2}$, $\cdots$, $X_{n}$ $(n > 1)$ 独立同分布,其方差 $\sigma^{2} > 0$,记 $\overline{X_{k}} = \frac{1}{k} \* \sum_{i=1}^{k} X_{i}$ $(1 \leq k \leq n)$,则 $\Cov\bigl( \overline{X_{s}},\overline{X_{t}} \bigr)$ $(1 \leq s,t \leq n)$ 的值等于 \kuo.

		\twoch{$\frac{\sigma^{2}}{\max\{ s,t \}}$}{$\frac{\sigma^{2}}{\min\{ s,t \}}$}{$\sigma^{2} \cdot \max\{s,t\}$}{$\sigma^{2} \cdot \min\{s,t\}$}
	\end{titwo}

	\begin{titwo}
		设随机变量 $X$ 的概率密度为
		\[
		f(x) = \begin{cases}
			\frac{3}{8}x^{2}, & 0 < x < 2, \\
			0, & \text{其他},
		\end{cases}
		\]
		则 $E\bigl( \frac{1}{X^{2}} \bigr) = $ \htwo.
	\end{titwo}

	\begin{titwo}
		设随机变量 $Y$ 服从参数为 $1$ 的指数分布,记
		\[
			X_{k} = \begin{cases}
				0, & Y \leq k, \\
				1, & Y > k,
			\end{cases} k = 1,2,
		\]
		则 $E(X_{1} + X_{2}) = $ \htwo.
	\end{titwo}

	\begin{titwo}
		已知离散型随机变量 $X$ 服从参数为 $2$ 的泊松分布,即 $P\{ X = k \} = \frac{ 2^{k}\ee^{-2} }{k!}$, $k = 0$, $1$, $2$, $\cdots$,则随机变量 $Z = 3X - 2$ 的数学期望 $EZ = $ \htwo.
	\end{titwo}

	\begin{titwo}
		设随机变量 $X_{1}$, $X_{2}$, $\cdots$, $X_{100}$ 独立同分布,且 $EX_{i} = 0$, $DX_{i} = 10$, $i = 1$, $2$, $\cdots$, $100$,令 $\overline{X} = \frac{1}{100} \* \sum_{i=1}^{100} X_{i}$,则 $E\Biggl[ \sum_{i=1}^{100} \bigl( X_{i} - \overline{X} \bigr)^{2} \Biggr] = $ \htwo.
	\end{titwo}

	\begin{titwo}
		设随机变量 $X$ 和 $Y$ 均服从 $B\bigl( 1,\frac{1}{2} \bigr)$,且 $D(X + Y) = 1$,则 $X$ 与 $Y$ 的相关系数 $\rho = $ \htwo.
	\end{titwo}

	\begin{titwo}
		已知随机变量 $X \sim N(-3,1)$, $Y \sim N(2,1)$,且 $X$, $Y$ 相互独立,设随机变量 $Z = X - 2Y + 7$,则 $Z \sim $ \htwo.
	\end{titwo}

	\begin{titwo}
		设相互独立的两个随机变量 $X$, $Y$ 具有同一分布律,且 $X$ 的分布律为
		\begin{center}
			\begin{tabular}{c|cc}
				\hline
				$X$ & $0$ & $1$ \\
				\hline
				$P$ & $\frac{1}{2}$ & $\frac{1}{2}$ \\
				\hline
			\end{tabular}
		\end{center}
		则随机变量 $Z = \max\{X,Y\}$ 的分布律为 \htwo.
	\end{titwo}

	\begin{titwo}
		设二维随机变量 $(X,Y)$ 的概率密度为
		\[
			f(x,y) = \begin{cases}
				\frac{1}{8} (x + y), & 0 \leq x \leq 2,0 \leq y \leq 2, \\
				0, & \text{其他}.
			\end{cases}
		\]
		则随机变量 $U = X + 2Y$, $V = -X$ 的协方差 $\Cov(U,$ $V) = $ \htwo.
	\end{titwo}

	\begin{titwo}
		一台设备由三个部件构成,在设备运转中各部件需要调整的概率分别为 $0.10$, $0.20$, $0.30$,设备部件状态相互独立,以 $X$ 表示同时需要调整的部件数,则 $X$ 的方差为 \htwo.
	\end{titwo}

	\begin{titwo}
		设 $(X,Y)$ 的概率密度为
		\[
		f(x,y) = \begin{cases}
			1, & 0 \leq |y| \leq x \leq 1, \\
			0, & \text{其他},
		\end{cases}
		\]
		则 $\Cov(X,Y) = $ \htwo.
	\end{titwo}

	\begin{titwo}
		若 $X_{1}$, $X_{2}$, $X_{3}$ 两两不相关,且 $DX_{i} = 1$ $(i = 1,2,$ $3)$,则 $D(X_{1} + X_{2} + X_{3}) = $ \htwo.
	\end{titwo}

	\begin{titwo}
		设随机变量 $X_{1}$, $X_{2}$, $X_{3}$ 相互独立,且 $X_{1} \sim B \bigl( 4,$ $\frac{1}{2} \bigr)$, $X_{2} \sim B \bigl( 6,\frac{1}{3} \bigr)$, $X_{3} \sim B\bigl( 6,\frac{1}{5} \bigr)$,则 $E[ X_{1} \* (X_{1} + X_{2} - X_{3}) ] = $ \htwo.
	\end{titwo}

	\begin{titwo}
		设随机变量 $X$ 与 $Y$ 的分布律为 \begin{tabular}{c|cc}
			\hline
			$X$ & $0$ & $1$ \\
			\hline
			$P$ & $\frac{1}{4}$ & $\frac{3}{4}$ \\
			\hline
		\end{tabular} 与 \begin{tabular}{c|cc}
			\hline
			$Y$ & $0$ & $1$ \\
			\hline
			$P$ & $\frac{1}{2}$ & $\frac{1}{2}$ \\
			\hline
		\end{tabular} 且相关系数 $\rho_{XY} = \frac{\sqrt{3}}{3}$,则 $(X,Y)$ 的分布律为 \htwo.
	\end{titwo}

	\begin{titwo}
		设二维随机变量 $(X,Y)$ 的分布律为
		\begin{center}
			\begin{tabular}{c|ccc}
				\hline
				\diagbox{$X$}{$Y$} & $-1$ & $0$ & $1$ \\
				\hline
				$-5$ & $0$ & $\frac{1}{9}$ & $\frac{1}{3}$ \\
				$-1$ & $\frac{1}{9}$ & $0$ & $\frac{2}{9}$ \\
				$1$ & $\frac{1}{9}$ & $\frac{1}{9}$ & $0$ \\
				\hline
			\end{tabular}
		\end{center}
		则 $X$ 与 $Y$ 的协方差为 \htwo.
	\end{titwo}

	\begin{titwo}
		设二维随机变量 $(X,Y)$ 的概率密度为
		\[
			f(x,y) = \begin{cases}
				3x, & 0 < x < 1,0 < y < x, \\
				0, & \text{其他},
			\end{cases}
		\]
		则随机变量 $Z = X - Y$ 的方差为 \htwo.
	\end{titwo}

	\begin{titwo}
		设总体 $X$ 和 $Y$ 相互独立,且分别服从正态分布 $N(0,4)$ 和 $N(0,7)$, $X_{1}$, $X_{2}$, $\cdots$, $X_{8}$ 和 $Y_{1}$, $Y_{2}$, $\cdots$, $Y_{14}$ 分别来自总体 $X$ 和 $Y$ 的简单随机样本,则统计量 $\bigl| \overline{X} - \overline{Y} \bigr|$ 的数学期望和方差分别为 \htwo.
	\end{titwo}

	\begin{titwo}
		假设一设备在任何长为 $t$ 的时间段内发生故障的次数 $N(t)$ 服从参数为 $\lambda t$ 的泊松分布 $(\lambda > 0)$,设两次故障之间时间间隔为 $T$,则 $ET = $ \htwo.
	\end{titwo}

	\begin{titwo}
		二维正态分布一般表示为 $N \bigl( \mu_{1},\allowbreak \mu_{2};\allowbreak \sigma_{1}^{2},\allowbreak \sigma_{2}^{2};\allowbreak \rho \bigr)$,设 $(X,Y) \sim N(1,1;4,9;0.5)$,令 $Z = 2X - Y$,则 $Z$ 与 $Y$ 的相关系数为 \htwo.
	\end{titwo}

	\begin{titwo}
		设随机变量 $X$ 的概率密度为
		\[
			f(x) = \begin{cases}
				ax, & 0 < x < 2, \\
				cx + b, & 2 \leq x \leq 4, \\
				0, & \text{其他},
			\end{cases}
		\]
		已知 $EX = 2$, $P\{1 < X < 3\} = \frac{3}{4}$,求:
		\begin{enumerate}
			\item $a$, $b$, $c$ 的值;
			\item 随机变量 $Y = \ee^{X}$ 的数学期望和方差.
		\end{enumerate}
	\end{titwo}

	\begin{titwo}
		袋中有 $n$ 张卡片,分别记有号码 $1$, $2$, $\cdots$, $n$,从中有放回地抽取 $k$ 次,每次抽取 $1$ 张,以 $X$ 表示所得号码之和,求 $EX$, $DX$.
	\end{titwo}

	\begin{titwo}
		设随机变量 $U$ 在 $[-2,2]$ 上服从均匀分布,记随机变量
		\[
			X = \begin{cases}
				-1, & U \leq -1, \\
				1, & U > -1,
			\end{cases}
			Y = \begin{cases}
				-1, & U \leq 1, \\
				1, & U > 1,
			\end{cases}
		\]
		求:
		\begin{enumerate}
			\item $\Cov(X,Y)$,并判定 $X$ 与 $Y$ 的独立性;
			\item $D[X(1 + Y)]$.
		\end{enumerate}
	\end{titwo}

	\begin{titwo}
		设试验成功的概率为 $\frac{3}{4}$,失败的概率为 $\frac{1}{4}$,独立重复试验直到成功两次为止,试求试验次数的数学期望.
	\end{titwo}

	\begin{titwo}
		设随机变量服从几何分布,其分布律为
		\[
			P\{ X = k \} = (1 - p)^{k - 1} p, 0 < p < 1, k = 1,2,\cdots,
		\]
		求 $EX$ 与 $DX$.
	\end{titwo}

	\begin{titwo}
		设 $(X,Y)$ 的概率密度为
		\[
			f(x,y) = \begin{cases}
				4xy \ee^{-(x^{2} + y^{2})}, & x > 0, y > 0, \\
				0, & \text{其他},
			\end{cases}
		\]
		求 $Z = \sqrt{X^{2} + Y^{2}}$ 的数学期望.
	\end{titwo}

	\begin{titwo}
		设连续型随机变量 $X$ 的所有可能取值在区间 $[a,b]$ 之内,证明:
		\begin{enumerate}
			\item $a \leq EX \leq b$;
			\item $DX \leq \frac{(b-a)^{2}}{4}$.
		\end{enumerate}
	\end{titwo}

	\begin{titwo}
		$\triangle ABC$ 边 $AB$ 上的高 $CD$ 长度为 $h$. 向 $\triangle ABC$ 中随机投掷一点 $P$,求
		\begin{enumerate}
			\item 点 $P$ 到边 $AB$ 的距离 $X$ 的概率密度;
			\item $X$ 的期望与方差.
		\end{enumerate}
	\end{titwo}

	\begin{titwo}
		对三台仪器进行检验,各台仪器产生故障的概率分别为 $p_{1}$, $p_{2}$, $p_{3}$,各台仪器是否产生故障相互独立,求产生故障仪器的台数 $X$ 的数学期望和方差.
	\end{titwo}

	\begin{titwo}
		一商店经销某种商品,每周进货量 $X$ 与顾客对该种商品的需求量 $Y$ 是相互独立的随机变量,且都服从区间 $[10,20]$ 上的均匀分布. 商店每售出一单位商品可得利润 $1000$ 元; 若需求量超过了进货量,商店可从其他商店调剂供应,这时每单位商品可得利润 $500$ 元,试计算此商店经销该种商品每周所得利润的期望值.
	\end{titwo}

	\begin{titwo}
		设 $X$, $Y$, $Z$ 是三个两两不相关的随机变量,数学期望全为零,方差都是 $1$,求 $X - Y$ 和 $Y - Z$ 的相关系数.
	\end{titwo}

	\begin{titwo}
		设二维随机变量 $(X,Y)$ 的概率密度为
		\[
			f(x,y) = \begin{cases}
				2 - x - y, & 0 < x < 1,0 < y < 1, \\
				0, & \text{其他},
			\end{cases}
		\]
		求:
		\begin{enumerate}
			\item 方差 $D(XY)$;
			\item 协方差 $\Cov(3X + Y, X - 2Y)$.
		\end{enumerate}
	\end{titwo}

	\begin{titwo}
		设 $X$ 的概率密度为
		\[
			f(x) = \begin{cases}
				\frac{1}{4}, & |x| < 1, \\
				\frac{1}{8}, & 1 \leq |x| \leq 3, \\
				0, & \text{其他}.
			\end{cases}
		\]
		令
		\[
			Y = g(X) = \begin{cases}
				X^{2} + 1, & X < 1, \\
				2, & X \geq 1,
			\end{cases}
		\]
		求:
		\begin{enumerate}
			\item $F_{Y}(y)$;
			\item $\Cov(X,Y)$.
		\end{enumerate}
	\end{titwo}

	\begin{titwo}
		设二维随机变量 $(U,V)$ 的概率密度为
		\[
			f(u,v) = \begin{cases}
				1, & 0 < u < 1, 0 < v < 2u, \\
				0, & \text{其他}.
			\end{cases}
		\]
		又设 $X$ 与 $Y$ 都是离散型随机变量,其中 $X$ 只取 $-1$, $0$, $1$ 三个值,$Y$ 只取 $-1$, $1$ 两个值,且 $EX = 0.2$, $EY = 0.4$. 又
		\begin{align*}
			P\{X = -1,Y = 1\} &= P\{X = 1,Y = -1\}\\
												&= P\{X = 0,Y = 1\}\\
												&= \frac{1}{3} P\Biggl\{ V \leq \frac{1}{2} \Biggl| U \leq \frac{1}{2} \Biggr\}.
		\end{align*}
		求:
		\begin{enumerate}
			\item $(X,Y)$ 的概率分布;
			\item $\Cov(X,Y)$.
		\end{enumerate}
	\end{titwo}

	\begin{titwo}
		产品寿命 $X$ 是一个随机变量,其分布函数与概率密度分别为 $F(x)$, $f(x)$. 产品已工作到时刻 $x$,在时刻 $x$ 后的单位时间 $\Delta x$ 内发生失效的概率称为产品在时刻 $x$ 的瞬时失效率,记为 $\lambda(x)$.
		\begin{enumerate}
			\item 证明 $\lambda(x) = \frac{f(x)}{1 - F(x)}$;
			\item 设某产品寿命的瞬时失效率函数为 $\lambda(x) = \alpha$,其中参数 $\alpha > 0$,求产品寿命 $X$ 的数学期望.
		\end{enumerate}
	\end{titwo}

	\begin{titwo}
		把一颗骰子独立地投掷 $n$ 次,记 $1$ 点出现的次数为随机变量 $X$,$6$ 点出现的次数为随机变量 $Y$,记
		\begin{align*}
		X_{i} &= \begin{cases}
			1, & \text{第 $i$ 次投掷出现 $1$ 点}, \\
			0, & \text{其他},
		\end{cases} \\
		Y_{j} &= \begin{cases}
			1, & \text{第 $j$ 次投掷出现 $6$ 点}, \\
			0, & \text{其他},
		\end{cases}
		\end{align*}
		$i$, $j = 1$, $2$, $\cdots$, $n$.
		\begin{enumerate}
			\item 求 $EX$, $DX$;
			\item 分别求 $i \ne j$ 时与 $i = j$ 时 $E(X_{i} Y_{j})$ 的值;
			\item 求 $X$ 与 $Y$ 的相关系数.
		\end{enumerate}
	\end{titwo}

	\begin{titwo}
		商店销售某种季节性商品,每售出一件获利 $500$ 元,季度末未售出的商品每件亏损 $100$ 元,以 $X$ 表示该季节此种商品的需求量,若 $X$ 服从正态分布 $N(100,4)$,问:
		\begin{enumerate}
			\item 进货量最少为多少时才能以超过 \SI{95}{\percent} 的概率保证供应?
			\item 进货量为多少时商店获利的期望值最大?
		\end{enumerate}
		($\varPhi(1.65) = 0.95$, $\varPhi(0.95) = 0.83$,其中 $\varPhi(x)$ 为标准正态分布函数)
	\end{titwo}

	\begin{titwo}
		设 $X$ 和 $Y$ 相互独立且均服从 0-1 分布,$P\{X = 1\} = P\{Y = 1\} = 0.6$. 试证明:$U = X + Y$, $V = X - Y$ 不相关且不独立.
	\end{titwo}

	\begin{titwo}
		对于任意两个事件 $A_{1}$, $A_{2}$,考虑随机变量
		\[
			X_{i} = \begin{cases}
				1, & \text{若事件 $A_{i}$ 出现}, \\
				0, & \text{若事件 $A_{i}$ 不出现}
			\end{cases}(i = 1,2).
		\]
		试证:随机变量 $X_{1}$ 和 $X_{2}$ 独立的充分必要条件是事件 $A_{1}$ 和 $A_{2}$ 相互独立.
	\end{titwo}

	\begin{titwo}
		某商品一周的需求量 $X$ 是随机变量,已知其概率密度为 $f(x) = \begin{cases}
			x\ee^{-x}, & x > 0, \\
			0, & \text{其他}.
		\end{cases}$ 假设各周的需求量相互独立,以 $U_{k}$ 表示 $k$ 周的总需求量,试求:
		\begin{enumerate}
			\item $U_{2}$ 和 $U_{3}$ 的概率密度 $f_{k}(x)$ $(k = 2,3)$;
			\item 接连三周中的周最大需求量的概率密度 $f_{(3)}(x)$.
		\end{enumerate}
	\end{titwo}

	\begin{titwo}
		假设 $G = \bigl\{ (x,y) | x^{2} + y^{2} \leq r^{2} \bigr\}$,而随机变量 $X$ 和 $Y$ 的联合分布是在区域 $G$ 上的均匀分布. 试确定随机变量 $X$ 和 $Y$ 的独立性和相关性.
	\end{titwo}