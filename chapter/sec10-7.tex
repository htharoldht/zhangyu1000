\section{方程组}

	\begin{titwo}
		设 $\bm \alpha_{1},\bm \alpha_{2},\bm \alpha_{3}$ 均为线性方程组 $\bm A \bm x = \bm b$ 的解,则下列向量
		\[
			\bm \alpha_{1} - \bm \alpha_{2},
			\bm \alpha_{1} - 2\bm \alpha_{2} + \bm \alpha_{3},
			\frac{1}{4}( \bm \alpha_{1} - \bm \alpha_{3} ),
			\bm \alpha_{1} + 3 \bm \alpha_{2} - 4 \bm \alpha_{3},
		\]
		其中是相应的齐次方程组 $\bm A \bm x = \bm 0$ 的解向量的个数为\kuo.

		\fourch{$4$}{$3$}{$2$}{$1$}
	\end{titwo}

	\begin{titwo}
		设 $\bm A$ 是秩为 $n - 1$ 的 $n$ 阶矩阵,$\bm \alpha_{1},\bm \alpha_{2}$ 是方程组 $\bm A \bm x = \bm 0$ 的两个不同的解向量,$k$ 是任意常数,则 $\bm A \bm x = \bm 0$ 的通解必定是\kuo.

		\twoch{$\bm \alpha_{1} + \bm \alpha_{2}$}{$k \bm \alpha_{1}$}{$k ( \bm \alpha_{1} + \bm \alpha_{2} )$}{$k ( \bm \alpha_{1} - \bm \alpha_{2} )$}
	\end{titwo}

	\begin{titwo}
		齐次线性方程组的系数矩阵 $\bm A_{4 \times 5} = \bigl[ \bm \beta_{1}, \bm \beta_{2},$ $\bm \beta_{3}, \bm \beta_{4}, \bm \beta_{5} \bigr]$ 经过初等行变换化成阶梯形矩阵为
		\[
			\bm A = \bigl[ \bm \beta_{1}, \bm \beta_{2}, \bm \beta_{3}, \bm \beta_{4}, \bm \beta_{5} \bigr] \xrightarrow{\text{初等行变换}} \begin{bsmallmatrix}
				1 & 2 & -1 & 5 & 2 \\
				0 & 1 & 2 & 6 & 0 \\
				0 & 0 & 0 & 4 & 0 \\
				0 & 0 & 0 & 0 & 0
			\end{bsmallmatrix},
		\]
		则\kuo.

		\onech{$\bm \beta_{1}$ 不能由 $\bm \beta_{3},\bm \beta_{4},\bm \beta_{5}$ 线性表出}{$\bm \beta_{2}$ 不能由 $\bm \beta_{1},\bm \beta_{3},\bm \beta_{5}$ 线性表出}{$\bm \beta_{3}$ 不能由 $\bm \beta_{1},\bm \beta_{2},\bm \beta_{5}$ 线性表出}{$\bm \beta_{4}$ 不能由 $\bm \beta_{1},\bm \beta_{2},\bm \beta_{3}$ 线性表出}
	\end{titwo}

	\begin{titwo}
		设 $\bm A$ 为 $m \times n$ 矩阵,则齐次线性方程组 $\bm A \bm X = \bm 0$ 仅有零解的充分条件是\kuo.

		\twoch{$\bm A$ 的列向量线性无关}{$\bm A$ 的列向量线性相关}{$\bm A$ 的行向量线性无关}{$\bm A$ 的行向量线性相关}
	\end{titwo}

	\begin{titwo}
		已知 $\bm \beta_{1},\bm \beta_{2}$ 是 $\bm A \bm X = \bm b$ 的两个不同的解,$\bm \alpha_{1},\bm \alpha_{2}$ 是相应的齐次方程组 $\bm A \bm X = \bm 0$ 的基础解系,$k_{1},k_{2}$ 是任意常数,则 $\bm A \bm X = \bm b$ 的通解是\kuo.

		\onech%
		{$k_{1} \bm \alpha_{1} + k_{2} ( \bm \alpha_{1} + \bm \alpha_{2} ) + \frac{\bm \beta_{1} - \bm \beta_{2}}{2}$}%
		{$k_{1} \bm \alpha_{1} + k_{2} ( \bm \alpha_{1} - \bm \alpha_{2} ) + \frac{\bm \beta_{1} + \bm \beta_{2}}{2}$}%
		{$k_{1} \bm \alpha_{1} + k_{2} ( \bm \beta_{1} - \bm \beta_{2} ) + \frac{\bm \beta_{1} - \bm \beta_{2}}{2}$}%
		{$k_{1} \bm \alpha_{1} + k_{2} ( \bm \beta_{1} - \bm \beta_{2} ) + \frac{\bm \beta_{1} + \bm \beta_{2}}{2}$}
	\end{titwo}

	\begin{titwo}
		设 $\bm \alpha_{1},\bm \alpha_{2},\bm \alpha_{3}$ 是四元非齐次线性方程组 $\bm A \bm X = \bm b$ 的三个解向量,且 $r(\bm A) = 3,\bm \alpha_{1}  = [1,2,3,4]^{\TT},\bm \alpha_{2} + \bm \alpha_{3} = [0,1,2,3]^{\TT},k$ 是任意常数,则方程组 $\bm A \bm X = \bm b$ 的通解是\kuo.

		\twoch%
		{$\begin{bsmallmatrix}
			1 \\
			2 \\
			3 \\
			4
		\end{bsmallmatrix} + k \begin{bsmallmatrix}
			1 \\
			1 \\
			1 \\
			1
		\end{bsmallmatrix}$}
		{$\begin{bsmallmatrix}
			1 \\
			2 \\
			3 \\
			4
		\end{bsmallmatrix} + k \begin{bsmallmatrix}
			0 \\
			1 \\
			2 \\
			3
		\end{bsmallmatrix}$}%
		{$\begin{bsmallmatrix}
			1 \\
			2 \\
			3 \\
			4
		\end{bsmallmatrix} + k \begin{bsmallmatrix}
			2 \\
			3 \\
			4 \\
			5
		\end{bsmallmatrix}$}%
		{$\begin{bsmallmatrix}
			1 \\
			2 \\
			3 \\
			4
		\end{bsmallmatrix} + k \begin{bsmallmatrix}
			3 \\
			4 \\
			5 \\
			6
		\end{bsmallmatrix}$}
	\end{titwo}

	\begin{titwo}
		设 $\bm A = \begin{bsmallmatrix}
			1 & 1 & a \\
			1 & a & 1 \\
			a & 1 & 1 \\
			2 & a + 1 & a + 3
		\end{bsmallmatrix},\bm B$ 是 $3$ 阶非零矩阵,且 $\bm A \bm B = \bm O$,则 $\bm A \bm x = \bm 0$ 的通解是\htwo.
	\end{titwo}

	\begin{titwo}
		求齐次线性方程组 $\begin{cases}
			x_{1} + x_{2} + x_{5} = 0,\\
			x_{1} + x_{2} - x_{3} = 0,\\
			x_{3} + x_{4} + x_{5} = 0
		\end{cases}$ 的基础解系.
	\end{titwo}

	\begin{titwo}
		已知 $r(\bm A) = r_{1}$,且方程组 $\bm A \bm X = \bm \alpha$ 有解,$r(\bm B) = r_{2}$,且 $\bm B \bm Y = \bm \beta$ 无解,设
		\[
			\bm A = [\bm \alpha_{1},\bm \alpha_{2},\cdots,\bm \alpha_{n}], \bm B = \bigl[\bm \beta_{1},\bm \beta_{2},\cdots,\bm \beta_{n}\bigr],
		\]
		且 $r\bigl(\bm \alpha_{1},\bm \alpha_{2},\cdots,\bm \alpha_{n},\bm \alpha,\bm \beta_{1},\bm \beta_{2},\cdots,\bm \beta_{n},\bm \beta\bigr) = r$,则\kuo.

		\twoch{$r = r_{1} + r_{2}$}{$r > r_{1} + r_{2}$}{$r = r_{1} + r_{2} + 1$}{$r \leq r_{1} + r_{2} + 1$}
	\end{titwo}

	\begin{titwo}
		设 $\bm A$ 是 $m \times n$ 矩阵,则方程组 $\bm A \bm X = \bm b$ 有唯一解的充分必要条件是\kuo.

		\onech{$m = n$ 且 $|\bm A| \ne 0$}{$\bm A \bm X = \bm 0$ 有唯一零解}{$\bm A$ 的列向量组 $\bm \alpha_{1},\bm \alpha_{2},\cdots,\bm \alpha_{n}$ 和 $\bm \alpha_{1},\bm \alpha_{2},\cdots,\bm \alpha_{n},\bm b$ 是等价向量组}{$r(\bm A) = n$,$\bm b$ 可由 $\bm A$ 的列向量线性表出}
	\end{titwo}

	\begin{titwo}
		设 $\bm A$ 是 $4 \times 5$ 矩阵,且 $\bm A$ 的行向量组线性无关,则下列说法不正确的是\kuo.
		
		\onech{$\bm A^{\TT} \bm X = \bm 0$ 只有零解}{$\bm A^{\TT} \bm A \bm X = \bm 0$ 必有无穷多解}{对任意的 $\bm b$,$\bm A^{\TT} \bm X = \bm b$ 有唯一解}{对任意的 $\bm b$,$\bm A \bm X = \bm b$ 有无穷多解}
	\end{titwo}

	\begin{titwo}
		已知 $n$ 阶矩阵 $\bm A$ 的各行元素之和均为零,且 $r(\bm A) = n - 1$,则线性方程组 $\bm A \bm X = \bm 0$ 的通解是\htwo.
	\end{titwo}

	\begin{titwo}
		已知非齐次线性方程组
		\begin{equation}\label{eq:120}
			\bm A_{3 \times 4} \bm X = \bm b
		\end{equation}
		有通解
		\[
			k_{1} [1,2,0,-2]^{\TT} + k_{2} [4,-1,-1,-1]^{\TT} + [1,0,-1,1]^{\TT},
		\]
		则满足方程组~\eqref{eq:120} 且满足条件 $x_{1} = x_{2},x_{3} = x_{4}$ 的解是\htwo.
	\end{titwo}

	\begin{titwo}
		已知 $4$ 阶方阵 $\bm A = [\bm \alpha_{1},\bm \alpha_{2},\bm \alpha_{3},\bm \alpha_{4}],$ $\bm \alpha_{1},$ $\bm \alpha_{2},$ $\bm \alpha_{3},$ $\bm \alpha_{4}$ 均为 $4$ 维列向量,其中 $\bm \alpha_{1},\bm \alpha_{2}$ 线性无关,若
		\begin{align*}
			\bm \beta &= \bm \alpha_{1} + 2 \bm \alpha_{2} - \bm \alpha_{3} \\
			&= \bm \alpha_{1} + \bm \alpha_{2} + \bm \alpha_{3} + \bm \alpha_{4} \\
			&= \bm \alpha_{1} + 3 \bm \alpha_{2} + \bm \alpha_{3} + 2 \bm \alpha_{4},
		\end{align*}
		则 $\bm A \bm x = \bm \beta$ 的通解为\htwo.
	\end{titwo}

	\begin{titwo}
		与 $\bm \alpha_{1} = [1,2,3,-1]^{\TT}, \bm \alpha_{2} = [0,1,1,2]^{\TT}, \bm \alpha_{3} = [2,1,$ $3,0]^{\TT}$ 都正交的单位向量是\htwo.
	\end{titwo}

	\begin{titwo}
		设向量组 $\bm \alpha_{1},\bm \alpha_{2},\cdots,\bm \alpha_{t}$ 是齐次线性方程组 $\bm A \bm x = \bm 0$ 的一个基础解系,向量 $\bm \beta$ 不是方程组 $\bm A \bm x = \bm 0$ 的解,即 $\bm A \bm \beta \ne \bm 0$. 证明:向量组 $\bm \beta,\bm \beta + \bm \alpha_{1},\bm \beta + \bm \alpha_{2},\cdots,\bm \beta + \bm \alpha_{t}$ 线性无关. 
	\end{titwo}

	\begin{titwo}
		设向量组
		\begin{gather*}
			\bm \alpha_{1} = [a_{11},a_{21},\cdots,a_{n1}]^{\TT},\\
			\bm \alpha_{2} = [a_{12},a_{22},\cdots,a_{n2}]^{\TT},\\
			\cdots,\\
			\bm \alpha_{s} = [a_{1s},a_{2s},\cdots,a_{ns}]^{\TT}.
		\end{gather*}
		证明:向量组 $\bm \alpha_{1},\bm \alpha_{2},\cdots,\bm \alpha_{s}$ 线性相关(线性无关)的充要条件是齐次线性方程组
		\[
			\begin{cases}
				a_{11}x_{1} + a_{12}x_{2} + \cdots + a_{1s}x_{s} = 0,\\
				a_{21}x_{1} + a_{22}x_{2} + \cdots + a_{2s}x_{s} = 0,\\
				\cdots\cdots\\
				a_{n1}x_{1} + a_{n2}x_{2} + \cdots + a_{ns}x_{s} = 0
			\end{cases}
		\]
		有非零解(唯一零解).
	\end{titwo}

	\begin{titwo}
		求下述线性方程组的解空间的维数:
		\[
			\begin{cases}
				x_{1} + 2x_{2} - 2x_{3} + 2x_{4} - x_{5} = 0,\\
				x_{1} + 2x_{2} - x_{3} + 3x_{4} - 2x_{5} = 0,\\
				2x_{1} + 4x_{2} - 7x_{3} + x_{4} + x_{5} = 0.
			\end{cases}
		\]
		并判断 $\bm \xi_{1} = [9,-1,2,-1,1]^{\TT}$ 是否属于该解空间.
	\end{titwo}

	\begin{titwo}
		已知线性方程组
		\[
			\begin{cases}
				a_{11}x_{1} + a_{12}x_{2} + a_{13}x_{3} + a_{14}x_{4} = a_{15},\\
				a_{21}x_{1} + a_{22}x_{2} + a_{23}x_{3} + a_{24}x_{4} = a_{25},\\
				a_{31}x_{1} + a_{32}x_{2} + a_{33}x_{3} + a_{34}x_{4} = a_{35},\\
				a_{41}x_{1} + a_{42}x_{2} + a_{43}x_{3} + a_{44}x_{4} = a_{45}
			\end{cases}
		\]
		的通解为 $[2,1,0,1]^{\TT} + k [1,-1,2,0]^{\TT}$. 记
		\[
			\bm \alpha_{j} = [a_{1j},a_{2j},a_{3j},a_{4j}]^{\TT}, j = 1,2,\cdots,5.
		\]
		问:
		\begin{enumerate}
			\item $\bm \alpha_{4}$ 能否由 $\bm \alpha_{1},\bm \alpha_{2},\bm \alpha_{3},\bm \alpha_{5}$ 线性表出,说明理由;
			\item $\bm \alpha_{4}$ 能否由 $\bm \alpha_{1},\bm \alpha_{2},\bm \alpha_{3}$ 线性表出,说明理由.
		\end{enumerate}
	\end{titwo}

	\begin{titwo}
		已知 $4$ 阶方阵 $\bm A = [\bm \alpha_{1},\bm \alpha_{2},\bm \alpha_{3},\bm \alpha_{4}],\bm \alpha_{1},\bm \alpha_{2},$ $\bm \alpha_{3},$ $\bm \alpha_{4}$ 均为 $4$ 维列向量,其中 $\bm \alpha_{2},\bm \alpha_{3},\bm \alpha_{4}$ 线性无关,$\bm \alpha_{1} = 2\bm \alpha_{2} - \bm \alpha_{3}$,如果 $\bm \beta = \bm \alpha_{1} + \bm \alpha_{2} + \bm \alpha_{3} + \bm \alpha_{4}$,求线性方程组 $\bm A \bm X = \bm \beta$ 的通解.
	\end{titwo}

	\begin{titwo}
		设 $\bm A_{m \times n}, r(\bm A) = m, \bm B_{n \times (n - m)}, r(\bm B) = n - m$,且满足关系式 $\bm A \bm B = \bm O$. 证明:若 $\bm \eta$ 是齐次线性方程组 $\bm A \bm X = \bm 0$ 的解,则必存在唯一的 $\bm \xi$,使得 $\bm B \bm \xi = \bm \eta$.
	\end{titwo}

	\begin{titwo}
		设三元非齐次线性方程组的系数矩阵 $\bm A$ 的秩为 $1$,已知 $\bm \eta_{1},\bm \eta_{2},\bm \eta_{3}$ 是它的三个解向量,且 $\bm \eta_{1} + \bm \eta_{2} = [1,2,3]^{\TT},\bm \eta_{2} + \bm \eta_{3} = [2,-1,1]^{\TT},\bm \eta_{3} + \bm \eta_{1} = [0,2,0]^{\TT}$,求该非齐次方程的通解.
	\end{titwo}

	\begin{titwo}
		设三元线性方程有通解
		\[
			k_{1} \begin{bsmallmatrix}
				-1\\
				3\\
				2
			\end{bsmallmatrix} + k_{2} \begin{bsmallmatrix}
				2\\
				-3\\
				1
			\end{bsmallmatrix} + \begin{bsmallmatrix}
				1\\
				-1\\
				3
			\end{bsmallmatrix},
		\]
		求原方程.
	\end{titwo}

	\begin{titwo}
		设 $\bm B$ 是秩为 $2$ 的 $5 \times 4$ 矩阵,$\bm \alpha_{1} = [1,$ $1,$ $2,$ $3]^{\TT}, \bm \alpha_{2} = [-1,1,4,-1]^{\TT}, \bm \alpha_{3} = [5,-1,-8,9]^{\TT}$ 是齐次线性方程组 $\bm B \bm x = \bm 0$ 的解向量,求 $\bm B \bm x = \bm 0$ 的解空间的一个标准正交基.
	\end{titwo}

	\begin{titwo}
		设 $\bm A$ 是 $3 \times 3$ 矩阵,$\bm \beta_{1},\bm \beta_{2},\bm \beta_{3}$ 是互不相同的 $3$ 维列向量,且都不是方程组 $\bm A \bm x = \bm 0$ 的解,记 $\bm B = \bigl[\bm \beta_{1},\bm \beta_{2},\bm \beta_{3}\bigr]$,且满足 $r(\bm A \bm B) < r(\bm A),r(\bm A \bm B) < r(\bm B)$. 则 $r(\bm A \bm B)$ 等于\kuo.
		
		\fourch{0}{1}{2}{3}
	\end{titwo}

	\begin{titwo}
		已知 $\bm \xi_{1},\bm \xi_{2},\cdots,\bm \xi_{r}(r \geq 3)$ 是 $\bm A \bm x = \bm 0$ 的基础解系,则下列向量组也是 $\bm A \bm x = \bm 0$ 的基础解系的是\kuo.
		\onech{$\bm \alpha_{1} = - \bm \xi_{2} - \bm \xi_{3} - \cdots - \bm \xi_{r},\bm \alpha_{2} = \bm \xi_{1} - \bm \xi_{3} - \bm \xi_{4} - \cdots - \bm \xi_{r},\bm \alpha_{3} = \bm \xi_{1} + \bm \xi_{2} - \bm \xi_{4} - \cdots - \bm \xi_{r},\cdots,\bm \alpha_{r} = \bm \xi_{1} + \bm \xi_{2} + \cdots + \bm \xi_{r-1}$}
		{$\bm \beta_{1} = \bm \xi_{2} + \bm \xi_{3} + \cdots + \bm \xi_{r},\bm \beta_{2} = \bm \xi_{1} + \bm \xi_{3} + \bm \xi_{4} + \cdots + \bm \xi_{r},\bm \beta_{3} = \bm \xi_{1} + \bm \xi_{2} + \bm \xi_{4} + \cdots + \bm \xi_{r},\cdots,\bm \beta_{r} = \bm \xi_{1} + \bm \xi_{2} + \cdots + \bm \xi_{r-1}$}%
		{$\bm \xi_{1},\bm \xi_{2},\cdots,\bm \xi_{r}$ 的一个等价向量组}%
		{$\bm \xi_{1},\bm \xi_{2},\cdots,\bm \xi_{r}$ 的一个等秩向量组}
	\end{titwo}

	\begin{titwo}
		设齐次线性方程组
		\[
			\begin{cases}
				a_{11} x_{1} + a_{12} x_{2} + a_{13} x_{3} + a_{14} x_{4} = 0, \\
				a_{21} x_{1} + a_{22} x_{2} + a_{23} x_{3} + a_{24} x_{4} = 0
			\end{cases}
		\]
		有基础解系 $\bm \beta_{1} = [b_{11},b_{12},b_{13},b_{14}]^{\TT},\bm \beta_{2} = [b_{21},\allowbreak b_{22},\allowbreak b_{23},\allowbreak ,\allowbreak b_{24}]^{\TT}$,记 $\bm \alpha_{1} = [a_{11},a_{12},a_{13},a_{14}]^{\TT},\bm \alpha_{2} = [a_{21},\allowbreak a_{22},\allowbreak a_{23},$ $a_{24}]^{\TT}$.

		证明:向量组 $\bm \alpha_{1},\bm \alpha_{2},\bm \beta_{1},\bm \beta_{2}$ 线性无关.
	\end{titwo}

	\begin{titwo}
		设 $\bm A$ 是 $3$ 阶矩阵,$\bm b = [9,18,-18]^{\TT}$,方程 $\bm A \bm x = \bm b$ 有通解 $k_{1} [-2,1,0]^{\TT} + k_{2} [2,0,1]^{\TT} + [1,2,-2]^{\TT}$,其中 $k_{1},k_{2}$ 是任意常数,求 $\bm A$ 及 $\bm A^{100}$.
	\end{titwo}

	\begin{titwo}
		已知线性方程组
		\[
			\begin{cases}
				b x_{1} - a x_{2} = -2 ab, \\
				-2 cx_{2} + 3 bx_{3} = bc, \\
				cx_{1} + ax_{3} = 0,
			\end{cases}
		\]
		则\kuo.
		
		\onech{当 $a,b,c$ 为任意实数时,方程组均有解}{当 $a = 0$ 时,方程组无解}{当 $b = 0$ 时,方程组无解}{当 $c = 0$ 时,方程组无解}
	\end{titwo}

	\begin{titwo}
		设 $a_{1},a_{2},\cdots,a_{n}$ 是互不相同的实数,且
		\[
			\bm A = \begin{bsmallmatrix}
				1 & a_{1} & a_{1}^{2} & \cdots & a_{1}^{n-1} \\
				1 & a_{2} & a_{2}^{2} & \cdots & a_{2}^{n-1} \\
				\vdots & \vdots & \vdots & & \vdots \\
				1 & a_{n} & a_{n}^{2} & \cdots & a_{n}^{n-1} \\
			\end{bsmallmatrix},
			\bm X = \begin{bsmallmatrix}
				x_{1} \\
				x_{2} \\
				\vdots \\
				x_{n}
			\end{bsmallmatrix},
			\bm b = \begin{bsmallmatrix}
				1 \\
				1 \\
				\vdots \\
				1
			\end{bsmallmatrix},
		\]
		求线性方程组 $\bm A \bm X = \bm b$ 的解.
	\end{titwo}

	\begin{titwo}
		问 $\lambda$ 为何值时,线性方程组
		\[
			\begin{cases}
				x_{1} + x_{3} = \lambda, \\
				4x_{1} + x_{2} + 2x_{3} = \lambda + 2, \\
				6x_{1} + x_{2} + 4x_{3} = 2 \lambda + 3
			\end{cases}
		\]
		有解,并求出解的一般形式.
	\end{titwo}

	\begin{titwo}
		问 $\lambda$ 为何值时,方程组
		\[
			\begin{cases}
				2x_{1} + \lambda x_{2} - x_{3} = 1, \\
				\lambda x_{1} - x_{2} + x_{3} = 2, \\
				4x_{1} + 5x_{2} - 5x_{3} = -1
			\end{cases}
		\]
		无解,有唯一解或有无穷多解?并在有无穷多解时写出方程组的通解.
	\end{titwo}

	\begin{titwo}
		已知线性方程组
		\[
			\begin{cases}
				x_{1} + x_{2} + x_{3} + x_{4} + x_{5} = a, \\
				3x_{1} + 2x_{2} + x_{3} + x_{4} - 3x_{5} = 0, \\
				x_{2} + 2x_{3} + 2x_{4} + 6x_{5} = b, \\
				5x_{1} + 4x_{2} + 3x_{3} + 3x_{4} - x_{5} = 2.
			\end{cases}
		\]
		\begin{enumerate}
			\item $a,b$ 为何值时,方程组有解;
			\item 方程组有解时,求方程组的导出组的基础解系;
			\item 方程组有解时,求方程组的全部解.
		\end{enumerate}
	\end{titwo}

	\begin{titwo}
		齐次线性方程组
		\[
			\begin{cases}
				\lambda x_{1} + x_{2} + \lambda^{2} x_{3} = 0, \\
				x_{1} + \lambda x_{2} + x_{3} = 0, \\
				x_{1} + x_{2} + \lambda x_{3} = 0
			\end{cases}
		\]
		的系数矩阵为 $\bm A$,若存在 $3$ 阶矩阵 $\bm B \ne \bm O$,使得 $\bm A \bm B = \bm O$,则 \kuo

		\twoch{$\lambda = -2$ 且 $|\bm B| = 0$}{$\lambda = -2$ 且 $|\bm B| \ne 0$}{$\lambda = 1$ 且 $|\bm B| = 0$}{$\lambda = 1$ 且 $|\bm B| \ne 0$}
	\end{titwo}

	\begin{titwo}
		方程组
		\[
			\begin{cases}
				x_{1} - x_{2} = a_{1}, \\
				x_{2} - x_{3} = a_{2}, \\
				x_{3} - x_{4} = a_{3}, \\
				x_{4} - x_{5} = a_{4}, \\
				x_{5} - x_{1} = a_{5}
			\end{cases}
		\]
		有解的充要条件是 \htwo.
	\end{titwo}

	\begin{titwo}
		设线性方程组
		\[
			\begin{cases}
				x_{1} - x_{2} - x_{3} - x_{4} = \lambda x_{1}, \\
				-x_{1} + x_{2} - x_{3} - x_{4} = \lambda x_{2}, \\
				-x_{1} - x_{2} + x_{3} - x_{4} = \lambda x_{3}, \\
				-x_{1} - x_{2} - x_{3} + x_{4} = \lambda x_{4}.
			\end{cases}
		\]
		则当 $\lambda$ 为何值时,方程组有解,有解时,求出所有的解.
	\end{titwo}

	\begin{titwo}
		已知 $\bm \eta_{1} = [-3,2,0]^{\TT}, \bm \eta_{2} = [-1,0,-2]^{\TT}$ 是线性方程组
		\[
			\begin{cases}
				a x_{1} + b x_{2} + c x_{3} = 2, \\
				x_{1} + 2x_{2} - x_{3} = 1, \\
				2x_{1} + x_{2} + x_{3} = -4
			\end{cases}
		\]
		的两个解向量,试求方程组的通解,并确定参数 $a,b,c$.
	\end{titwo}

	\begin{titwo}
		已知方程组 (\Rmnum{1})
		\[
			\begin{cases}
				x_{1} + x_{4} = 1, \\
				x_{2} - 2x_{4} = 2, \\
				x_{3} + x_{4} = -1
			\end{cases}
		\]
		与方程组 (\Rmnum{2})
		\[
			\begin{cases}
				-2x_{1} + x_{2} + ax_{3} - 5x_{4} = 1, \\
				x_{1} + x_{2} - x_{3} + bx_{4} = 4, \\
				3x_{1} + x_{2} + x_{3} + 2x_{4} = c
			\end{cases}
		\]
		是同解方程组,试确定参数 $a,b,c$.
	\end{titwo}

	\begin{titwo}
		设方程组
		\[
			\begin{cases}
				x_{1} + x_{2} - x_{3} + 2x_{4} - x_{5} = 1, \\
				x_{1} - x_{2} + 5x_{4} + ax_{5} = b, \\
				2x_{1} + 4x_{2} - 3x_{3} + x_{4} - 4x_{5} = 1, \\
				x_{1} + 3x_{2} - x_{3} - x_{4} = -1, \\
				2x_{1} - x_{3} + ax_{4} - 4x_{5} = -1.
			\end{cases}
		\]
		问:
		\begin{enumerate}
			\item $a,b$ 为何值时,方程组有唯一解;
			\item $a,b$ 为何值时,方程组无解;
			\item $a,b$ 为何值时,方程组有无穷多解,并求其通解.
		\end{enumerate}
	\end{titwo}

	\begin{titwo}
		已知方程组 (\Rmnum{1}) $\begin{cases}
			x_{1} + 3x_{2} - 3x_{4} = 1,\\
			-7x_{2} + 3x_{3} + x_{4} = -3
		\end{cases}$ 及方程组 (\Rmnum{2}) 的通解为
		\[
			k_{1} [-1,1,1,0]^{\TT} + k_{2} [2,-1,0,1]^{\TT} + [-2,-3,0,0]^{\TT}.
		\]
		求方程组 (\Rmnum{1}), (\Rmnum{2}) 的公共解.
	\end{titwo}

	\begin{titwo}
		设 $\bm A$ 是 $n$ 阶矩阵,对于齐次线性方程组 (\Rmnum{1})~$\bm A^{n} \bm x = \bm 0$ 和 (\Rmnum{2}) $\bm A^{n + 1} \bm x = \bm 0$,现有命题\\
		\circled{1}(\Rmnum{1}) 的解必是 (\Rmnum{2}) 的解;\\
		\circled{2}(\Rmnum{2}) 的解必是 (\Rmnum{1}) 的解;\\
		\circled{3}(\Rmnum{1}) 的解不一定是 (\Rmnum{2}) 的解;\\
		\circled{4}(\Rmnum{2}) 的解不一定是 (\Rmnum{1}) 的解.\\
		其中正确的是 \kuo.

		\fourch{\circled{1}\circled{4}}{\circled{1}\circled{2}}{\circled{2}\circled{3}}{\circled{3}\circled{4}}
	\end{titwo}

	\begin{titwo}
		设 $\bm A$ 是 $n$ 阶实矩阵,则对线性方程组 (\Rmnum{1})~$\bm A \bm X = \bm 0$ 和 (\Rmnum{2})~$\bm A^{\TT} \bm A \bm X = \bm 0$,必有 \kuo.
		
		\onech{(\Rmnum{2}) 的解是 (\Rmnum{1}) 的解,(\Rmnum{1}) 的解也是 (\Rmnum{2}) 的解}%
		{(\Rmnum{2}) 的解是 (\Rmnum{1}) 的解,但 (\Rmnum{1}) 的解不是 (\Rmnum{2}) 的解}%
		{(\Rmnum{1}) 的解不是 (\Rmnum{2}) 的解,(\Rmnum{2}) 的解也不是 (\Rmnum{1}) 的解}%
		{(\Rmnum{1}) 的解是 (\Rmnum{2}) 的解,但 (\Rmnum{2}) 的解不是 (\Rmnum{1}) 的解}
	\end{titwo}

	\begin{titwo}
		设 $\bm A$ 是 $m \times s$ 矩阵,$\bm B$ 是 $s \times n$ 矩阵,则齐次线性方程组 $\bm B \bm X = \bm 0$ 和 $\bm A \bm B \bm X = \bm 0$ 是同解方程组的一个充分条件是 \kuo.

		\twoch{$r(\bm A) = m$}{$r(\bm A) = s$}{$r(\bm B) = s$}{$r(\bm B) = n$}
	\end{titwo}

	\begin{titwo}
		设四元齐次线性方程组 (\Rmnum{1}) 为 $\begin{cases}
			x_{1} + x_{2} = 0, \\
			x_{2} - x_{4} = 0,
		\end{cases}$ 又已知某齐次线性方程组 (\Rmnum{2}) 的通解为
		\[
			k_{1} [0,1,1,0]^{\TT} + k_{2} [-1,2,2,1]^{\TT}.
		\]
		\begin{enumerate}
			\item 求线性方程组 (\Rmnum{1}) 的基础解系;
			\item 问线性方程组 (\Rmnum{1}) 和 (\Rmnum{2}) 是否有非零公共解?若有,则求出所有的非零公共解. 若没有,则说明理由.
		\end{enumerate}
	\end{titwo}

	\begin{titwo}
		已知齐次线性方程组 (\Rmnum{1}) 的基础解系为 $\bm \xi_{1} = [1,0,1,1]^{\TT}$, $\bm \xi_{2} = [2,1,0,-1]^{\TT}$, $\bm \xi_{3} = [0,2,1,-1]^{\TT}$,添加两个方程
		\[
			\begin{cases}
				x_{1} + x_{2} + x_{3} + x_{4} = 0, \\
				x_{1} + 2x_{2} + 2x_{4} = 0
			\end{cases}
		\]
		后组成齐次线性方程组 (\Rmnum{2}),求 (\Rmnum{2}) 的基础解系.
	\end{titwo}

	\begin{titwo}
		已知线性方程组 (\Rmnum{1}) \[\begin{cases}
			3x_{1} - x_{2} + 8x_{3} + x_{4} = 0, \\
			x_{1} + 3x_{2} - 9x_{3} + 7x_{4} = 0
		\end{cases}\] 及线性方程组 (\Rmnum{2}) 的基础解系
		\[
			\bm \xi_{1} = [-3,7,2,0]^{\TT}, \bm \xi_{2} = [-1,-2,0,1]^{\TT}.
		\]
		求方程组 (\Rmnum{1}) 和 (\Rmnum{2}) 的公共解.
	\end{titwo}

	\begin{titwo}
        已知齐次线性方程组 (\Rmnum{1}) 为
        \[
            \begin{cases}
                x_{1} + x_{2} - x_{3} = 0, \\
                x_{2} + x_{3} - x_{4} = 0,
            \end{cases}
        \]
        齐次线性方程组 (\Rmnum{2}) 的基础解系为
		\[
			\bm \xi_{1} = [-1,1,2,4]^{\TT}, \bm \xi_{2} = [1,0,1,1]^{\TT}.
		\]
		\begin{enumerate}
			\item 求方程组 (\Rmnum{1}) 的基础解系;
			\item 求方程组 (\Rmnum{1}) 与 (\Rmnum{2}) 的全部非零公共解,并将非零公共解分别由方程组 (\Rmnum{1}), (\Rmnum{2}) 的基础解系线性表示.
		\end{enumerate}
	\end{titwo}

	\begin{titwo}
		设 $\bm A$ 是 $4$ 阶方阵,则下列线性方程组是同解方程组的是 \kuo.

		\twoch{$\bm A \bm x = \bm 0$; $\bm A^{2} \bm x = \bm 0$}{$\bm A^{2} \bm x = \bm 0$; $\bm A^{3} \bm x = \bm 0$}{$\bm A^{3} \bm x = \bm 0$; $\bm A^{4} \bm x = \bm 0$}{$\bm A^{4} \bm x = \bm 0$; $\bm A^{5} \bm x = \bm 0$}
	\end{titwo}

	\begin{titwo}
		设线性方程组
		\begin{equation}\label{eq:156.1}
			\begin{cases}
				x_{1} + 3x_{3} + 5x_{4} = 0, \\
				x_{1} - x_{2} - 2x_{3} + 2x_{4} = 0, \\
				2x_{1} - x_{2} + x_{3} + 3x_{4} = 0,
			\end{cases}
		\end{equation}
		添加一个方程 $a x_{1} + 2x_{2} + b x_{3} - 5x_{4} = 0$ 后,成为方程组
		\begin{equation}\label{eq:156.2}
			\begin{cases}
				x_{1} + 3x_{3} + 5x_{4} = 0, \\
				x_{1} - x_{2} - 2x_{3} + 2x_{4} = 0, \\
				2x_{1} - x_{2} + x_{3} + 3x_{4} = 0, \\
				ax_{1} + 2x_{2} + bx_{3} - 5x_{4} = 0.
			\end{cases}
		\end{equation}
		\begin{enumerate}
			\item 求方程组~\eqref{eq:156.1} 的通解;
			\item $a,b$ 满足什么条件时,\eqref{eq:156.1}~\eqref{eq:156.2} 是同解方程组.
		\end{enumerate}
	\end{titwo}